\section{XAFS Data Modeling}\label{sect:modeling}

In this section, we'll work through an example of structural refinement of
EXAFS. The FeO data shown and reduced in the previous section will be
analyzed here.  Of course, we know the expected results for this system,
but it will serve to demonstrate the principles of XAFS modeling, and allow
us to comment on a number of features and subtleties in data modeling.

FeO has a simple rock salt structure, with Fe surrounded by 6 O, with
octahedral symmetry, and then 12 Fe atoms in the next shell.  Starting with
this simple structure, we can calculate scattering amplitudes $f(k)$ and
phase-shifts, $\delta(k)$ theoretically. A complete description of this
calculation is beyond the scope of this treatment, but a few details will be
given below.

Once we have these theoretical scattering factors, we can use them in the
EXAFS equation to refine structural parameters from our data.  That is,
we'll use the calculated functions $f(k)$ and $\delta(k)$ (and also
$\lambda(k)$) in the EXAFS equation to predict the and modify the
structural parameters $R$, $N$, and $\sigma^2$ from
Eq.~\ref{Eq:xafs_final}, and also allow $E_0$ (that is, the energy for
which $k=0$) to change until we get the best-fit to the $\chi(k)$ of the
data.  Because of the availability of the Fourier transform, we actually
have a choice of doing the refinement with the measured $\chi(k)$ or with
the Fourier transformed data.  Because working in $R$-space allows us to
selectively ignore higher coordination shells, using $R$-space for the
fitting has several advantages and we will use it in the examples here.
When analyzing the data this way, the full complex XAFS $\chi(R)$, not just
the magnitude $|\chi(R)|$, must be used.  The examples shown here are done
with the {\feff}\cite{feff6} program to construct the theoretical factors,
and the {\ifeffit}\cite{ifeffit} package to do the analysis.  Some details
of these particular programs will be given, but similar results would be
obtained with any of several other EXAFS analysis tools.

\subsection{Running and Using FEFF for EXAFS calculations}

In order to calculate the $f(k)$ and $\delta(k)$ needed for the analysis,
the {\feff} program\cite{feff6} starts with a cluster of atoms, builds
atomic potentials from this, and simulates a photo-electron with a
particular energy being emitted by a particular absorbing atom and
propagating along a set of scattering paths\cite{feffit}.  {\feff}
represents a substantial work of modern theoretical condensed matter
physics, and includes many conceptually subtle but quantitatively important
effects, including include the finite size of the scattering atoms, and
many-body effects due to the fact that electrons are indistinguishable
particles that must satisfy Pauli's exclusion
principle\cite{RehrAlbersRMP2000}.  These are beyond the scope of this
work, but we mention them here because they have important consequences for
how we use {\feff}.

We do not, as may have been inferred from some of the earlier discussion,
use {\feff} to calculate $f(k)$ and $\delta(k)$ for the scattering of, say,
an oxygen atom, and use that for all scattering of oxygen.  Instead we use
{\feff} to calculate the EXAFS for a particular path, say Fe-O-Fe taken
from a realistic cluster of atoms.  This includes the rather complex
propagation of the photo-electron out of the Fe atom, through the sea of
electrons in a iron oxide material, scattering from an oxygen atom with
finite size, and propagating back to the absorbing Fe atom.  As a result of
this, we make rough but realistic simulations of the EXAFS with {\feff} for
a particular set of paths, and then {\emph{refine}} the path lengths and
coordination numbers for those paths.

Starting with a cluster of atoms (which does not need to be crystalline, but
this is often easy to use), {\feff} determines the important scattering
paths, and writes out a separate file for the scattering contributions to
from each scattering path.  Conveniently (and though it does not calculate
these factors individually), it breaks up the results in a way that can be
put into the standard form of the EXAFS equation, even for
multiple-scattering paths.  This allows analysis procedures to easily
refine distances, apply multiplicative factors for coordination numbers and
$S_0^2$,  and apply disorder terms.  Because the outputs are of a uniform
format, we can readily mix outputs from different runs of the programs,
which is important for modeling complex structures with multiple
coordination environments for the absorbing atom.

\subsection{First-shell fitting}

For an example of modeling EXAFS, we start with FeO, a transition metal
oxide with the particularly simple rock-salt structure, while still being
representative of many systems found in nature and studied by EXAFS, in
that the first shell is oxygen, and the second shell is a heavier metal
element.  We begin with modeling the first Fe-O shell of FeO, take a brief
diversion into  the meaning and interpretation of the statistical results
of the modeling, and then continue on to analyze the second shell.

\begin{Nfig}{width=3.5truein}{feo_1shell_chirmag}
  \caption{First shell fit to the EXAFS of FeO, showing the magnitude of
    the Fourier transform of the EXAFS,  $|\chi(R)|$, for data (blue) and
    best fit model (red).}
  \label{Fig:FIT:feo1a}
\end{Nfig}

We start with the crystalline structure and generate the input format for
{\feff}, run {\feff}, and gather the outputs.  For the rock-salt structure
of FeO with six Fe-O near-neighbors in octahedral coordination, and twelve
Fe-Fe second neighbors, there will be one file for the six Fe-O scattering
paths, and one file for the twelve Fe-Fe scattering paths.  To model the
first shell EXAFS, we use the simulation for the Fe-O scattering path, and
refine the values for $NS_0^2$, $R$, and $\sigma^2$.  We set $S_0^2$ to
0.75.  We also usually need to (and in this example, will) refine a value
for $E_0$, the threshold defining where $k$ is 0.  This is usually
necessary because the choice of $E_0$ from the maximum of the first
derivative of the spectra is {\emph{ad hoc}}, and because the choice of
energy threshold in the calculation is somewhat crude.  While the refined
value for $E_0$ may be small, we will see that $E_0$ is strongly correlated
with $R$, so that getting both its value and uncertainty is important.

\begin{table}[tbh]
  \caption{Best values and uncertainties (in parentheses) for the
    refined first shell parameters for FeO.  The refinement fit the
    components of $\chi(R)$ between $R=[0.9, 2.0]\,\text{\AA}$ after a Fourier
    transform using $k=[2.5, 13.5]\,\text{\AA}^{-1}$, a $k$-weight of 2, and a
    Hanning window function.  $S_0^2$ was fixed to 0.75.}
  \label{TABLE:FIT:feo1}
  \begin{center}
    \begin{tabular}{cccccc}
    Shell & ${N}$ & ${R}\, (\text{\AA})$ & ${\sigma^2}\, ({\text{\AA}^2})$ & ${\Delta E_0}$ (eV) \\
    \hline    \noalign{\smallskip}
    Fe-O  &  5.5(0.5) & 2.10(0.01) & 0.015(.002) & -3.2(1.0) \\
    \noalign{\smallskip}
    \hline
  \end{tabular}
\end{center}
\end{table}

The results of the initial refinement is shown in Figure~\ref{Fig:FIT:feo1a},
with best values and estimated uncertainties for the refined parameters
given in Table~\ref{TABLE:FIT:feo1}.  These values are not perfect for
crystalline FeO, especially in that the distance is contracted from the
expected value of $2.14\, \text{\AA}$, but are reasonably close for a first
analysis.


It is instructive to look at this refinement more closely, and discuss a
few of the details.  The refinement was done on the data in $R$-space,
after a Fourier transform.  This weighted $\chi(k)$ by $k^2$, and
multiplied it by a Hanning window function with a range between $k=[2.5,
13.5]\,\text{\AA}^{-1}$, with a $dk$ parameter of $2 \,\text{\AA}^{-1}$.  The
refinement used the real and imaginary components of $\chi(R)$ between
$R=[0.9, 2.0]\,\text{\AA}$.  The $k^2\chi(k)$ for the data and best-fit model,
as well as the window function are shown on the left side of Figure~\ref{Fig:FIT:feo1b}.

From this and Figure~\ref{Fig:FIT:feo1a}, it is evident that the higher frequency
components (that is, from the second shell of Fe-Fe) dominate $k^2\chi(k$).
This is a useful reminder of the power of the Fourier transform in XAFS
analysis: it allows us to concentrate on one shell at a time and ignore the
others, even if they have larger overall amplitude.

\begin{Sfig}{width=3truein}{feo_1shell_chik}{feo_1shell_chirreal}
\caption{EXAFS $k^2\chi(k)$ (left) for data (blue) and best-fit model (red)
  for the first shell of FeO, and the window function, $\Omega(k)$, used
  for the Fourier transform to $\chi(R)$.  While the red curve shows the
  best-fit to the 1st shell of the EXAFS, this is not obvious from looking
  at $k^2\chi(k)$.  The complex Fourier transform EXAFS $\chi(R)$ (right)
  for the real part (solid) and magnitude (dashed) of the data (blue) and
  best-fit model (red) in the region of the first and second shell shows
  the model matches the data very well for the first shell.}
  \label{Fig:FIT:feo1b}
\end{Sfig}

\subsection{Fit statistics and estimated uncertainties}


At this point, we should pause to discuss some details of the fit,
including the fit statistics and how the best-fit values and uncertainties
are determined in the refinement.  Because the EXAFS equation is complex,
and non-linear in the parameters we wish to refine, the refinement is done
with a non-linear least-squares fit.    Such a fit uses the standard statistical
definitions for chi-square  and least-squares  to determine the
best values for the set of  parameters varied as those that find the
smallest possible sum of squares of the difference in the model and data.
The standard definition of the chi-square or $\chi^2$ statistic (note the
use of $\chi^2$ from standard statistical treatments -- don't confuse with
the EXAFS $\chi$!) that is minimized in a least-squares fit is defined as
\begin{equation}
  \chi^2  =  \sum_i^{N_{\rm data}} \frac{[y_i^{\rm data} - y_i^{\rm  model}({x})]^2}{\epsilon^2}
  \label{Eq:FIT:chisquare}
\end{equation}
\noindent where $y_i^{\rm data} $ is our experimental data, $y_i^{\rm
  model}({x})$ is the model which depends on the variable fitting
parameters $x$, $\epsilon$ is uncertainty in the data, and $N_{\rm data}$ is
the number of points being fit.  Each of
these terms deserves more discussion, and we will expand on these, while
striving for brevity.

The set of variable parameters $x$ are the values actually changed in the
fit.  If we had fixed a value (say, for $N$), it would not be a variable.
Below, we will impose relationships between parameters in the EXAFS
equation, for example, using a single variable to give the value for $E_0$ for
multiple paths.   This counts as one variable in the fit, even though it may
influence the value of several physical parameters for multiple paths.

Importantly, the $\chi^2$ definition does not actually specify what is
meant by the data $y$.  In the fit above, we used the real and imaginary
components of $\chi(R)$, after Fourier transforming the data with a
particular window function and $k$-weight.  Using different parameters for
the transform would result in different data (and model) to be fit, and
could change the results.  We could have tried to fit the $k^2\chi(k)$
without Fourier transforming, but as can be seen from
Fig.~\ref{Fig:FIT:feo1b}, the fit may have been substantially worse.  But,
as we are at liberty to decide what is meant by ``the data'' to be modeled,
we can include using multiple spectra that we wish to model with one set of
parameters $x$.  Of course, any transformation or extensions we make to the
data must be applied equally to the model for the data.   In general, we
find that fitting EXAFS data in $R$-space strikes a good balance between
not changing the data substantially, and allowing us to select the $k$ and
$R$ ranges we wish to and are able to model.

The uncertainty in the data is represented by $\epsilon$ in the above
definition for $\chi^2$.  Of course, this too must match what we mean by
``the data'', and will generally mean the uncertainty in $\chi(R)$ in the
range of the data we're modeling.  There are many general strategies for
estimating uncertainties in data, usually based on involved statistical
treatment of many measurements.  Such efforts are very useful, but tend to
be challenging to apply for every EXAFS measurement.  A convenient if crude
approach is to rely on the fact that EXAFS decays rapidly with $R$ and to
assert that the data at very high $R$ (say, above $15\, \text{\AA}$) reflects
the noise level.  Applying this to the $R$ range of our data assumes that
the noise is independent of $R$ (white noise), which is surely an
approximation.  The advantage of the approach is that it can be applied
automatically for any set of data.  Tests have shown that it gives a
reasonable estimate for data of low to normal quality, and underestimates
the noise level for very good data.  A simple relationship based on
Parseval's theorem and Fourier analysis can be used to relate $\epsilon_R$,
the noise estimate in $\chi(R)$ to $\epsilon_k$, the noise in $\chi(k)$\cite{NewvilleBoyanov}.

There are a couple additional statistics that are particularly useful\cite{StandCrit}.  One
of these is reduced chi-square, defined as $\chi^2_{\nu} = \chi^2 / (N_{\rm
  data} - N_{\rm varys})$ where $N_{\rm varys}$ is the number of variable
parameters in the fit.  This has the feature of being a measure of
goodness-of-fit that is takes into account the number of variables used.
In principle, for a good fit and data with well-characterized
uncertainties, $\epsilon$, this value should approach 1.  $\chi^2_{\nu}$ is
especially useful when comparing whether one fit is better than another.  In
simplest terms, a fit with a lower value for $\chi^2_{\nu}$ is said to be
better than one with a higher value.  Of course, there is some statistical
uncertainty in this assertion, and confidence intervals and $F$ tests can
be applied to do a more rigorous analysis.  For EXAFS analysis, a principle
difficulty is that the values of $\chi^2_{\nu}$ are often several orders of
magnitude worse than 1, far worse than can be ascribed to a poor estimate
of $\epsilon$.  Partly because of this, another statistic is $\cal{R}$,
defined as
\begin{equation}
  {\cal{R}} = \frac{\sum_i^{N_{\rm data}}[y_i^{\rm data} -
    yi^{\rm model}({x})]^2 }{
    \sum_i^{N_{\rm data}} [{y_i^{\rm data}}]^2}
  \label{Eq:FIT:rfactor}
\end{equation}
\noindent
which gives the size of the misfit relative to the norm of the data.  This
value is typically found to below 0.05 or so for good fits, and is often
found to be much better than that.

Last, and possibly most surprising, we discuss the problem of identifying
$N_{\rm data}$.  When measuring $\mu(E)$ we are free to sample as many
energy points as we wish, but increasing the number of points in $\mu(E)$
doesn't necessarily mean we have a better measure of the first shell EXAFS.
In the previous chapter, we mentioned that the zero-padding and fine
spacing of $k$ data sets the spacing of data in $R$.  We should be clear
that this can (and usually does) greatly over-sample the data in $R$ space.


For any waveform or signal, the Nyquist-Shannon sampling theorem tells us
that the maximum $R$ that can be measured is related to the spacing of
sample in $k$, according to (for EXAFS, with conjugate Fourier variables of
$k$ and $2R$)
\begin{equation}
  R_{\rm max} = \frac{\pi}{2\delta k}
  \label{Eq:FIT:rmax}
\end{equation}
\noindent
where $R_{\rm max}$ is the maximum $R$ value we can detect, and $\delta k$
is the spacing for the $\chi(k)$ data.  Using $\delta k =
0.05\,\text{\AA}^{-1}$ is common in EXAFS, which means we cannot detect EXAFS
contributions beyond $31.4\, \text{\AA}$.  As the converse of this, the
resolution for an EXAFS spectrum -- the separation in $R$ below which two
peaks can be independently measured -- is given as
\begin{equation}
  \delta R = \frac{\pi}{2k_{\rm max}}
  \label{Eq:FIT:rres}
\end{equation}
\noindent
where $k_{\rm max}$ is the maximum measured value of $k$.  In short, what
matters most for determining how well $\chi(R)$ is measured the signal at
any particular value of $R$ is how many oscillations we have in $\chi(k)$.

Related to both $R_{\rm max}$ and the resolution $\delta R$, and also
resulting from basic signal processing theory and Fourier analysis, the
number of independent measurements in a band-limited waveform is
\begin{equation}
  N_{\rm ind} \approx \frac{2 \Delta k \Delta R}{\pi} + 1
  \label{Eq:FIT:nidp}
\end{equation}
\noindent
where $\Delta k$ and $\Delta R$ are the range of useful data in $k$ and
$R$.  For completeness, the above equation is often given with a ``+2''
instead of a ``+1''\cite{Stern93} in the EXAFS literature, though we will
follow the more conservative estimate, and note that it would give an upper
limit on the number of variables that could be determined from a set of
data.  In any event, making finer measurements of $\mu(E)$ and $\chi(k$)
might allow us to reliably see data at higher $R$ values, but it does
provide finer resolution of the distances well below the Nyquist cut off
frequency, $R_{\rm max}$.  In order to be able to extract more information
for a particular range of $R$, data to higher $k$ is needed.

Thus, we should  modify  the  definition of $\chi^2$ (and $\chi^2_{\nu}$)
used to reflect the number of truly independent data points in the data, as
\begin{equation}
  \chi^2  = \frac{N_{\rm ind}}{N_{\rm data}} \sum_i^{N_{\rm data}}
  \frac{[y_i^{\rm data} - y_i^{\rm  model}({x})]^2}{\epsilon^2}
  \label{Eq:FIT:chisquare_nidp}
\end{equation}
\noindent where $N_{\rm ind}$ is given by Eq.~\ref{Eq:FIT:nidp} and $N_{\rm
  data}$ is the number of samples used for the data, even if this far
exceeds $N_{\rm ind}$ .  Values of $N_{\rm ind}$ for real EXAFS data is not
very large.  In the first shell fit to FeO, we used $k=[2.5,
13.5]\,\text{\AA}^{-1}$ and $R=[0.9, 2.0]\,\text{\AA}$ which gives $N_{\rm ind}
\approx 8.7$, and we used 4 variables in the fit -- roughly half the
maximum.  For higher shells and more complicated structures, we will have
to come up with ways to limit the number of variables in fits.

By measuring how $\chi^2$ changes as variables are moved way from their
best-fit values, estimates of the uncertainties for variables and
correlations between pairs of variables can be determined.  Standard
statistical arguments indicate that 1-$\sigma$ error bars (that indicate a
68\% confidence in the value) should increase $\chi^2$ by 1 from its
best-fit value.  This assumes that $\chi^2_{\nu} \approx 1$, which is
usually not true for EXAFS data.  As a consequence, it it common in the
EXAFS literature to report uncertainties for values that increase $\chi^2$
by $\chi^2_{\nu}$.  This is equivalent to asserting that a fit is actually
good, and scaling $\epsilon$ so that $\chi^2_{\nu}$ is 1.

Estimating uncertainties and correlations between variables can be very
fast, as the computational algorithms for minimization compute and use
intermediate values related to these (in the form of the covariance matrix)
to find the best values. Uncertainties determined this way include the
effect of correlations (that is, moving the value for one variable away
from its best value may change what the best value for another variable
would be), but also make some assumptions about how the values of the
variable interact.  More sophisticated approaches, including brute-force
exploration of values by stepping a variable through a set of
values and repeatedly refining the rest of the variables, can give better
measures of uncertainties, but are more computationally expensive.

Though the aim of a fit is to find the best values for the fitting
parameters $x$, the computational techniques used do no guarantee that the
``global'' minimum of $\chi^2$ is found, only that a ``local'' minima is
found based on the starting values.  This, of course, can cause
considerable concern.  Care should be taken to check that the results found
are not too sensitive to the starting values for the variables or data
manipulation parameters including Fourier transform ranges and weights, and
background subtraction parameters.  Checking for false global minima is
somewhat more involved.  Fortunately, for EXAFS analysis with reasonably
well-defined shells, false minima usually give obviously non-physical
results, such as negative or huge coordination number, negative values for
$\sigma^2$.  Another warning sign for a poor model is an $E_0$ shift away
from the maximum of the first derivative by 10 eV or more.  This can
sometimes happen, but can also indicate that the model $\chi(k)$ has jumped
a half or whole period away from its correct position, and that the
amplitude parameters may be very far off, as if the $Z$ for the scatterer
is wrong.

Our diversion into fitting statistics is complete, and we can return to our
first shell fit to Fe-O before continuing on.  The data was estimated to
have a $\epsilon_R \approx 5\times10^{-3}$ and $\epsilon_k \approx
2\times10^{-4}$, which is a typical noise level for experimental $\chi(k)$ data.
With a standard $k$ grid of $0.05\,\text{\AA}^{-1}$, and an $R$ grid of
$\approx 0.0307\, \text{\AA}$, the fit had 72 data points, but $N_{\rm ind}
\approx 8.7$.  Scaled to $N_{\rm ind}$ as in
Eq.~\ref{Eq:FIT:chisquare_nidp}, the fit has $\chi^2 \approx 243$ and
$\chi^2_{\nu} \approx 51.7$ (again with 4 variables), and ${\cal{R}}
\approx 0.005$.


\subsection{Second-shell fitting}

\begin{Sfig}{width=3truein}{feo_2shell_chirmag}{feo_2shell_chirreal}
\caption{EXAFS $|\chi(R)|$ (left) and ${\rm Re}[\chi(R)]$ (right)
  for FeO (blue) and best-fit model (red) for the
  first two shells around Fe, including Fe-O and Fe-Feo scattering paths.}
  \label{Fig:FIT:feo2a}
\end{Sfig}

We are now ready to include the second shell in the model for the FeO
EXAFS.  To do this, we simply add the path for Fe-Fe scattering to the sum
in the EXAFS equation.  We will add variables for $R$, $N$, and $\sigma^2$
for the Fe-Fe shell to those for the Fe-O scattering path.  We'll use the
same value for $E_0$ for both the Fe-O and Fe-Fe path, and keep all
parameters  the same as for the fit above, except that we'll extend the $R$
range to be $R=[0.9, 3.1]\,\text{\AA}$.   This will increase  $N_{\rm ind}$ to
$\approx 15.7$, while we've increased the number of variables to 7.



\begin{table}[tbh]
  \caption{Best values and uncertainties (in parentheses) for the
    refined first (Fe-O) and second (Fe-Fe) shells  for FeO.  The refinement fit the
    components of $\chi(R)$ between $R=[0.9, 3.0]\,\text{\AA}$ with all other
    parameters as in Table~\ref{TABLE:FIT:feo1}.}
  \label{TABLE:FIT:feo2}
  \begin{center}
    \begin{tabular}{cccccc}
    Shell & ${N}$ & ${R}\, (\text{\AA})$ & $\sigma^2\, (\text{\AA}^2)$ & ${\Delta E_0}$ (eV) \\
    \hline    \noalign{\smallskip}
    Fe-O   &   5.3(0.5)  &   2.11(0.01) & 0.013(.002) & -1.2(0.5)\\
    Fe-Fe  &  13.4(1.3) & 3.08(0.01) & 0.015(.001) &  -1.2(0.5)\\
    \noalign{\smallskip}
    \hline
  \end{tabular}
\end{center}
\end{table}

\begin{Sfig}{width=3truein}{feo_2shell_chik}{feo_2shell_chirpaths}
  \caption{Contributions of the first and second shell to the total model
    fit to the Feo EXAFS. On the left, the fit (red) matches the data
    (blue) much better than in Figure~\ref{Fig:FIT:feo1b}.  Note that,
    compared to the Fe-O contribution the Fe-Fe contribution has a shorter
    period corresponding to longer interatomic distance, and has magnitude
    centered at higher $k$, as predicted by the $f(k)$ function shown in
    Figure~\ref{Fig:THE:scatt}.   On the right, the  $|\chi(R)|$  of the
    contributions from the two shells is shown.   Though there is a sharp
    dip a 2 {\AA} between  peaks for the two shells, there is substantial
    leakage from one shell to another.}
  \label{Fig:FIT:feo2b}
\end{Sfig}

The fit is shown in Figure~\ref{Fig:FIT:feo2a} and values and uncertainties
for the fitted variables are given in Table~\ref{TABLE:FIT:feo2}.  The fit
gave fit statistics of $\chi^2 \approx 837$, $\chi^2_\nu \approx 96$, and
${\cal{R}} \approx 0.0059$.  The structural values for distances and
coordination number are consistent with the known crystal structure of FeO,
though the Fe-O distance is a bit shorter than expected, and the Fe-Fe is a
bit longer than expected, both suggesting that there may be some
contamination of a ferric iron phase in the sample.  The fits are shown in
Figure~\ref{Fig:FIT:feo2a}, and individual contributions to the total best-fit
spectrum are shown in both $k$- and $R$-space in Figure~\ref{Fig:FIT:feo2b}.
An important aspect of using fitting techniques to model experimental data
is the ability to compare different fits to decide which of two different
models is better.  We will illustrate this by questioning the assumption in
the above the model that the $E_0$ parameter should be exactly the same for
the Fe-O and Fe-Fe scattering path.  Changing this model to allow another
variable parameter and re-running the fit is straightforward.  For this
data set, the fit results are close enough to the previous fit that the
graphs of $\chi(k)$ and $\chi(R)$ are nearly unchanged.  The newly refined
values for the parameter are given in Table~\ref{TABLE:FIT:feo3}.  Compared
to the values in Table~\ref{TABLE:FIT:feo2}, the results are very similar
except for the values of $E_0$ and a slight increase in uncertainties.

\begin{table}[tbh]
  \caption{Best values and uncertainties (in parentheses) for the
    refined first (Fe-O) and second (Fe-Fe) shells  for FeO for a model
    just like that shown in Table~\ref{TABLE:FIT:feo2} except that the 2
    values for $E_0$ are allowed to vary independently.}
  \label{TABLE:FIT:feo3}
  \begin{center}
    \begin{tabular}{cccccc}
    Shell & ${N}$ & ${R}\, (\text{\AA})$ & $\sigma^2\, (\text{\AA}^2)$ & ${\Delta E_0}$ (eV) \\
    \hline    \noalign{\smallskip}
    Fe-O   &   5.3(0.6)  &   2.12(0.01) & 0.013(.002) & -0.7(1.2)\\
    Fe-Fe  &  13.3(1.3) & 3.08(0.01) & 0.015(.001) &  -1.5(0.8)\\
    \noalign{\smallskip}
    \hline
  \end{tabular}
\end{center}
\end{table}


The fit statistics for this refinement are $\chi^2 \approx 811$,
$\chi^2_\nu \approx 105$, and ${\cal{R}} \approx 0.0057$.  Since both
$\chi^2$ and $\cal{R}$ have decreased, the model with 2 independent $E_0$
values is clearly a closer match to the data.  But we added a variable to
the model, so it is reasonable to expect that the fit should be better.
But is it better enough to justify the additional variable?  The simplest
approach to answering this question is to ask if $\chi^2_\nu$ has improved.
In this case, it has not -- it went from roughly 96 to 105.  Since these
statistics all have uncertainties associated with them, a slightly more
subtle question is: what is the probability that the second fit is better
than the first?  A standard statistical $F$-test can be used to give this
probability, which turns out to be about 32\% for these two fits (that is,
with $N_{\rm ind}=15.7$, $\chi^2 \approx 837$ for 8 variables and $\chi^2
\approx 811$ for 7 variables).

Another way to look at this is to ask if the added variable ($E_0$ for the
Fe-Fe shell) found a value that was significantly different from the value
it would have otherwise had.  The two values for $E_0$ in the ``2 $E_0$
model'' are noticeably different from one another -- approximately at the
limits of their uncertainties -- but both are consistent with the value
found in the ``1 $E_0$ model''.  This also leads us to the conclusion that
the additional variable $E_0$ is not actually necessary for modeling this
data.

We've seen that structural refinement of EXAFS data can be somewhat
complicated, even for a relatively straightforward system such as FeO.
Many real systems can be much more challenging, but the fundamental
principles described here remain the same.  The ability to alter which of
the physical parameters describing the different paths in the EXAFS sum are
independently varied in the refinement, and test the robustness of these,
can be especially important for more sophisticated analysis.  One way to
think about this is that in the first version of the above example, we used
the value of one variable for two different path variables -- $E_0$ for the
Fe-O and Fe-Fe paths, and then demonstrated that using one value for these
two physical parameters was robust.  That would be the simplest possible
type of placing constraints on an EXAFS analysis, and it had the noticeable
advantage of improving the fit because it used fewer independent variables.
For a mixed coordination shell, perhaps a mixture of Fe-O and Fe-S, one may
want to include paths for Fe-O and Fe-S and ask the model not to simply
refine the weight of each of these independently but rather to ask what
fraction of the Fe atoms are coordinated by oxygen.  To do this, one would
vary the fraction $x_{\rm FeO}$ as a pre-factor to the amplitude term for
the Fe-O path and constrain the coordination number for the Fe-S path to
use $1 - x_{\rm FeO}$.  More complex constraints can be imposed when
simultaneously refining data from different edges or different temperatures
measured on the same sample.  In a sense, the use of multiple paths for
different parts of the $R$ range for $\chi(R)$ in the fit above is merely
the starting point for thinking about how different contributions can be
put together to make a model for a set of data.

The basic formalism for modeling EXAFS data has been given, based on the
Path expansion, theoretical calculations of the contributions for these
paths, the Fourier transform, and a statistical understanding of the
information present in a real EXAFS spectrum.  We have illustrated a simple
approach to refining a structural model using EXAFS data, and used
statistical methods to compare two different candidate models.  Finally we
have outlined the route forward to building models for more complex EXAFS
data.

Two distinct and essential challenges exist for EXAFS analysis.  First, the
complexity of the theoretical calculations for photo-electron scattering in
make it difficult to get theoretical scattering factors $f(k)$ and
$\delta(k)$ that can match the accuracy of measured EXAFS data.  By itself,
this has proven to not be a serious problem, as the EXAFS literature is
full of examples showing the accuracy of the results from EXAFS despite the
imperfect theoretical calculations.  Second, the limited information
contained in a finite EXAFS spectrum coupled with the number of scattering
paths needed to model real systems makes building and testing realistic
models for complex systems challenging.  Progress in analytics tools for
EXAFS continues to make the building and testing of such models easier and
more robust, but it still requires a fair amount of expertise and care.
Despite the challenges, EXAFS has been proven to give reproducible and
reliable measures of the local structure around selected atoms that cannot
be obtained in any other way, and the number of scientists using EXAFS in
mature and new and exciting fields continues to grow.


