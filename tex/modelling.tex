\section{XAFS Data Modeling}\label{sect:modeling}

In this section, we'll work through an example of structural refinement of
EXAFS. The FeO data shown and reduced in the previous section will be
analyzed here.  Of course, we know the expected results for this system,
but it will serve to demonstrate the principles of XAFS modeling, and allow
us to comment on a number of features and subtleties in data modeling.

FeO has a simple rock salt structure, with Fe surrounded by 6 O, with
octahedral symmetry, and then 12 Fe atoms in the next shell.  Starting with
this simple structure, we can calculate scattering amplitudes $f(k)$ and
phase-shifts, $\delta(k)$ theoretically. A complete description of this
calculation is beyond the scope of this treatment, but a few details will be
given below.

Once we have these theoretical scattering factors, we can use them in the
EXAFS equation to refine structural parameters from our data.  That is,
we'll use the calculated functions $f(k)$ and $\delta(k)$ (and also
$\lambda(k)$) in the EXAFS equation to predict the and modify the
structural parameters $R$, $N$, and $\sigma^2$ from
Eq.~\ref{Eq:xafs_final}, and also allow $E_0$ (that is, the energy for
which $k=0$) to change until we get the best-fit to the $\chi(k)$ of the
data.  Because of the availability of the Fourier transform, we actually
have a choice of doing the refinement with the measured $\chi(k)$ or with
the Fourier transformed data.  Because working in $R$-space allows us to
selectively ignore higher coordination shells, using $R$-space for the
fitting has several advantages and we will use it in the examples here.
When analyzing the data this way, the full complex XAFS $\chi(R)$, not just
the magnitude $|\chi(R)|$, must be used.  The examples shown here are done
with the {\feff}\cite{feff6} program to construct the theoretical factors,
and the {\ifeffit}\cite{ifeffit} package to do the analysis.  Some details
of these particular programs will be given, but similar results would be
obtained with any of several other EXAFS analysis tools.

\subsection{Running and Using {\feff} for EXAFS calculations}

In order to calculate the $f(k)$ and $\delta(k)$ needed for the analysis,
the {\feff} program\cite{feff6} starts with a cluster of atoms, builds
atomic potentials from this, and simulates a photo-electron with a
particular energy being emitted by a particular absorbing atom and
propagating along a set of scattering paths\cite{feffit}.  {\feff}
represents a substantial work of modern theoretical condensed matter
physics, and includes many conceptually subtle but quantitatively important
effects, including include the finite size of the scattering atoms, and
many-body effects due to the fact that electrons are indistinguishable
particles that must satisfy Pauli's exclusion
principle\cite{RehrAlbersRMP2000}.  These are beyond the scope of this
work, but we mention them here because they have important consequences for
how we use {\feff}.

We do not, as may have been inferred from some of the earlier discussion,
use {\feff} to calculate $f(k)$ and $\delta(k)$ for the scattering of, say,
an oxygen atom, and use that for all scattering of oxygen.  Instead we use
{\feff} to calculate the EXAFS for a particular path, say Fe-O-Fe taken
from a realistic cluster of atoms.  This includes the rather complex
propagation of the photo-electron out of the Fe atom, through the sea of
electrons in a iron oxide material, scattering from an oxygen atom with
finite size, and propagating back to the absorbing Fe atom.  As a result of
this, we make rough but realistic simulations of the EXAFS with {\feff} for
a particular set of paths, and then {\emph{refine}} the path lengths and
coordination numbers for those paths.

Starting with a cluster of atoms (which does not need to be crystalline, but
this is often easy to use), {\feff} determines the important scattering
paths, and writes out a separate file for the scattering contributions to
from each scattering path.  Conveniently (and though it does not calculate
these factors individually), it breaks up the results in a way that can be
put into the standard form of the EXAFS equation, even for
multiple-scattering paths.  This allows analysis procedures to easily
refine distances, apply multiplicative factors for coordination numbers and
$S_0^2$,  and apply disorder terms.  Because the outputs are of a uniform
format, we can readily mix outputs from different runs of the programs,
which is important for modeling complex structures with multiple
coordination environments for the absorbing atom.

\subsection{First-shell fitting}

For an example of modeling EXAFS, we start with FeO, a transition metal
oxide with the particularly simple rock-salt structure, while still being
representative of many systems found in nature and studied by EXAFS, in
that the first shell is oxygen, and the second shell is a heavier metal
element.  We begin with modeling the first Fe-O shell of FeO, take a brief
diversion into  the meaning and interpretation of the statistical results
of the modeling, and then continue on to analyze the second shell.

\begin{Nfig}{width=3.5truein}{feo_1shell_chirmag}
  \caption{First shell fit to the EXAFS of FeO, showing the magnitude of
    the Fourier transform of the EXAFS,  $|\chi(R)|$, for data (blue) and
    best fit model (red).}
  \label{Fig:FIT:feo1a}
\end{Nfig}

We start with the crystalline structure and generate the input format for
{\feff}, run {\feff}, and gather the outputs.  For the rock-salt structure
of FeO with six Fe-O near-neighbors in octahedral coordination, and twelve
Fe-Fe second neighbors, there will be one file for the six Fe-O scattering
paths, and one file for the twelve Fe-Fe scattering paths.  To model the
first shell EXAFS, we use the simulation for the Fe-O scattering path, and
refine the values for $NS_0^2$, $R$, and $\sigma^2$.  We set $S_0^2$ to
0.75.  We also usually need to (and in this example, will) refine a value
for $E_0$, the threshold defining where $k$ is 0.  This is usually
necessary because the choice of $E_0$ from the maximum of the first
derivative of the spectra is {\emph{ad hoc}}, and because the choice of
energy threshold in the calculation is somewhat crude.  While the refined
value for $E_0$ may be small, it is strongly correlated with $R$, so that
getting both its value and uncertainty is important.

\begin{table}[tbh]
  \caption{Best values and uncertainties (in parentheses) for the
    refined first shell parameters for FeO.  The refinement fit the
    components of $\chi(R)$ between $R=[0.9, 2.0]\,\text{\AA}$ after a Fourier
    transform using $k=[2.5, 13.5]\,\text{\AA}^{-1}$, a $k$-weight of 2, and a
    Hanning window function.  $S_0^2$ was fixed to 0.75.}
  \label{TABLE:FIT:feo1}
  \begin{center}
    \begin{tabular}{cccccc}
    Shell & ${N}$ & ${R}\, (\text{\AA})$ & ${\sigma^2}\, ({\text{\AA}^2})$ & ${\Delta E_0}$ (eV) \\
    \hline    \noalign{\smallskip}
    Fe-O  &  5.5(0.5) & 2.10(0.01) & 0.015(.002) & -3.2(1.0) \\
    \noalign{\smallskip}
    \hline
  \end{tabular}
\end{center}
\end{table}

The results of the initial refinement is shown in Figure~\ref{Fig:FIT:feo1a},
with best values and estimated uncertainties for the refined parameters
given in Table~\ref{TABLE:FIT:feo1}.  These values are not perfect for
crystalline FeO, especially in that the distance is contracted from the
expected value of $2.14\, \text{\AA}$, but are reasonably close for a first
analysis.


It is instructive to look at this refinement more closely, and discuss a
few of the details.  The refinement was done on the data in $R$-space,
after a Fourier transform.  This weighted $\chi(k)$ by $k^2$, and
multiplied it by a Hanning window function with a range between $k=[2.5,
13.5]\,\text{\AA}^{-1}$, with a $dk$ parameter of $2 \,\text{\AA}^{-1}$.  The
refinement used the real and imaginary components of $\chi(R)$ between
$R=[0.9, 2.0]\,\text{\AA}$.  The $k^2\chi(k)$ for the data and best-fit model,
as well as the window function are shown on the left side of Figure~\ref{Fig:FIT:feo1b}.

From this and Figure~\ref{Fig:FIT:feo1a}, it is evident that the higher frequency
components (that is, from the second shell of Fe-Fe) dominate $k^2\chi(k$).
This is a useful reminder of the power of the Fourier transform in XAFS
analysis: it allows us to concentrate on one shell at a time and ignore the
others, even if they have larger overall amplitude.

\begin{Sfig}{width=3truein}{feo_1shell_chik}{feo_1shell_chirreal}
\caption{EXAFS $k^2\chi(k)$ (left) for data (blue) and best-fit model (red)
  for the first shell of FeO, and the window function, $\Omega(k)$, used
  for the Fourier transform to $\chi(R)$.  While the red curve shows the
  best-fit to the 1st shell of the EXAFS, this is not obvious from looking
  at $k^2\chi(k)$.  The complex Fourier transform EXAFS $\chi(R)$ (right)
  for the real part (solid) and magnitude (dashed) of the data (blue) and
  best-fit model (red) shows
  the model matches the data very well for the first shell.}
  \label{Fig:FIT:feo1b}
\end{Sfig}

The fits done here use standard definitions of fit statistics, including
the chi-square or $\chi^2$ statistic (note the use of $\chi^2$ from
standard statistical treatments -- don't confuse with the EXAFS $\chi$!)
that is minimized in a least-squares fit is defined as
\begin{equation}
  \chi^2  =  \sum_i^{N_{\rm data}} \frac{[\chi_i^{\rm data} - \chi_i^{\rm  model}({x})]^2}{\epsilon^2}
  \label{Eq:FIT:chisquare}
\end{equation}
\noindent where $\chi_i^{\rm data} $ is our experimental data, $\chi_i^{\rm
  model}({x})$ is the model which depends on the variable fitting
parameters $x$, $\epsilon$ is uncertainty in the data, and $N_{\rm data}$
is the number of points being fit.  As the modeling is typically done after
a Fourier transform, the $\chi_i^{\rm data} $ is our experimental data
after a Fourier transform, and includes both real and imaginary components.

Because of the limited range of data in $k$ and $R$, EXAFS is limited in
the number of parameters that can be fit, with the number of independent
measurements available from an EXAFS spectrum being given by
\begin{equation}
  N_{\rm ind} \approx \frac{2 \Delta k \Delta R}{\pi} + 1
  \label{Eq:FIT:nidp}
\end{equation}
\noindent
where $\Delta k$ and $\Delta R$ are the range of useful data in $k$ and
$R$.  This value should be used as an upper limit on the number of
parameters that can be sensibly determined from EXAFS.  Of course, a fit
will also need to estimating uncertainties and correlations between
variables.  This can be readily done with standard statistical methods that
are fast and generally reliable.  The first shell fit to Fe-O had a noise
level in the data of $\epsilon_R \approx 5\times10^{-3}$ and $\epsilon_k
\approx 2\times10^{-4}$, which is a typical noise level for experimental
$\chi(k)$ data.  Values for other statistics were $N_{\rm ind} \approx 8.7$
and $\chi^2 \approx 243$.


\subsection{Second-shell fitting}

\begin{Sfig}{width=3truein}{feo_2shell_chirmag}{feo_2shell_chirreal}
\caption{EXAFS $|\chi(R)|$ (left) and ${\rm Re}[\chi(R)]$ (right)
  for FeO (blue) and best-fit model (red) for the
  first two shells around Fe, including Fe-O and Fe-Feo scattering paths.}
  \label{Fig:FIT:feo2a}
\end{Sfig}

\begin{Sfig}{width=3truein}{feo_2shell_chik}{feo_2shell_chirpaths}
  \caption{Contributions of the first and second shell to the total model
    fit to the Feo EXAFS. On the left, the fit (red) matches the data
    (blue) much better than in Figure~\ref{Fig:FIT:feo1b}.  Note that,
    compared to the Fe-O contribution the Fe-Fe contribution has a shorter
    period corresponding to longer interatomic distance, and has magnitude
    centered at higher $k$, as predicted by the $f(k)$ function shown in
    Figure~\ref{Fig:THE:scatt}.   On the right, the  $|\chi(R)|$  of the
    contributions from the two shells is shown.   Though there is a sharp
    dip a 2 {\AA} between  peaks for the two shells, there is substantial
    leakage from one shell to another.}
  \label{Fig:FIT:feo2b}
\end{Sfig}

To include the second shell in the model for the FeO EXAFS, we simply add
the path for Fe-Fe scattering to the sum in the EXAFS equation.  We will
add variables for $R$, $N$, and $\sigma^2$ for the Fe-Fe shell to those for
the Fe-O scattering path.  We'll use the same value for $E_0$ for both the
Fe-O and Fe-Fe path, and keep all parameters the same as for the fit above,
except that we'll extend the $R$ range to be $R=[0.9, 3.1]\,\text{\AA}$.
This will increase $N_{\rm ind}$ to $\approx 15.7$, while we've increased
the number of variables to 7.

\begin{table}[bh]
  \caption{Best values and uncertainties (in parentheses) for the
    refined first (Fe-O) and second (Fe-Fe) shells  for FeO.  The refinement fit the
    components of $\chi(R)$ between $R=[0.9, 3.0]\,\text{\AA}$ with all other
    parameters as in Table~\ref{TABLE:FIT:feo1}.}
  \label{TABLE:FIT:feo2}
  \begin{center}
    \begin{tabular}{cccccc}
    Shell & ${N}$ & ${R}\, (\text{\AA})$ & $\sigma^2\, (\text{\AA}^2)$ & ${\Delta E_0}$ (eV) \\
    \hline    \noalign{\smallskip}
    Fe-O   &   5.3(0.5)  &   2.11(0.01) & 0.013(.002) & -1.2(0.5)\\
    Fe-Fe  &  13.4(1.3) & 3.08(0.01) & 0.015(.001) &  -1.2(0.5)\\
    \noalign{\smallskip}
    \hline
  \end{tabular}
\end{center}
\end{table}


The fit is shown in Figure~\ref{Fig:FIT:feo2a} and values and uncertainties
for the fitted variables are given in Table~\ref{TABLE:FIT:feo2}.  The fit
gave fit statistics of $\chi^2 \approx 837$.  The structural values for
distances and coordination number are consistent with the known crystal
structure of FeO, though the Fe-O distance is a bit shorter than expected,
and the Fe-Fe is a bit longer than expected, both suggesting that there may
be some contamination of a ferric iron phase in the sample.  The fits are
shown in Figure~\ref{Fig:FIT:feo2a}, and individual contributions to the total
best-fit spectrum are shown in both $k$- and $R$-space in
Figure~\ref{Fig:FIT:feo2b}.

% The basic formalism for modeling EXAFS data has been given, based on the
% Path expansion, theoretical calculations of the contributions for these
% paths, the Fourier transform, and a statistical understanding of the
% information present in a real EXAFS spectrum.  We have illustrated a simple
% approach to refining a structural model using EXAFS data, and used
% statistical methods to compare two different candidate models.  Finally we
% have outlined the route forward to building models for more complex EXAFS
% data.

% Two distinct and essential challenges exist for EXAFS analysis.  First, the
% complexity of the theoretical calculations for photo-electron scattering in
% make it difficult to get theoretical scattering factors $f(k)$ and
% $\delta(k)$ that can match the accuracy of measured EXAFS data.  By itself,
% this has proven to not be a serious problem, as the EXAFS literature is
% full of examples showing the accuracy of the results from EXAFS despite the
% imperfect theoretical calculations.  Second, the limited information
% contained in a finite EXAFS spectrum coupled with the number of scattering
% paths needed to model real systems makes building and testing realistic
% models for complex systems challenging.  Progress in analytics tools for
% EXAFS continues to make the building and testing of such models easier and
% more robust, but it still requires a fair amount of expertise and care.



