\section{XAFS Data Reduction}

For all XAFS data, whether measured in transmission or fluorescence (or
electron emission), the data reduction and analysis are essentially the
same.  First, the measured intensity signals are converted to $\mu(E)$, and
then reduced to $\chi(k)$.  After this data reduction, $\chi(k)$ can be
analyzed and modeled using the XAFS equation.  In this section, we'll go
through the steps of data reduction, from measured intensities to
$\chi(k)$, which generally proceeds as:

\begin{enumerate}
\item Convert measured intensities to $\mu(E)$, possibly correcting
  systematic measurement errors such as self-absorption effects and detector
  deadtime.

\item Identify the threshold energy $E_0$, typically as the energy of the
  maximum derivative of $\mu(E)$.

\item Subtract a smooth pre-edge function from $\mu(E)$ to get rid of any
  instrumental background and absorption from other edges.

\item Determine the edge jump, $\Delta\mu$, at the threshold energy, and
  normalize $\mu(E)$, so that the pre-edge subtracted and normalized
  $\mu(E)$ goes from approximately 0 below the threshold energy to 1 well
  above the threshold energy.  This represents the absorption of a single
  X-ray, and is useful for XANES analysis.

\item Remove a smooth post-edge background function approximating
  $\mu_0(E)$, thereby isolating the XAFS $\chi = {(\mu-\mu_0)} /
  \Delta\mu$.

\item convert $\chi$ from energy $E$ to photo-electron wavenumber  $k =
  \sqrt{2m(E-E_0)/\hbar^2}$.

\item $k$-weight the XAFS $\chi(k)$ and Fourier transform into $R$-space
\end{enumerate}
\noindent
We'll go through each of these steps in slightly more detail, and show them
graphically using real XAFS data.

As with many things, the first step is often the most challenging.  Here,
the differences between measurements made in  transmission and fluorescence
mode are most pronounced.   For transmission measurements, we simply ignore
the sample thickness, and use
\begin{equation}
  \mu(E) =\ln (I_0/I)
\end{equation}
\noindent where $I_0$ and $I$ are the signals measured from the ion
chambers.  Typically, the signals measured as $I_0$ and $I$ are actually
integrated voltages over some predefined time where the voltages are taken
from the output of current amplifiers with input currents from ion chambers
as input.  Thus the measurements are not the incident flux in photons per
second.  Rather, they are scaled measures of the flux {\emph{absorbed}} in
the ion chamber.  For the most part, the difference between what we think
of as $I_0$ (incident X-ray flux, in photons/second) and what we actually
measure for $I_0$ (scaled, integrated current generated from X-rays
absorbed in the ion chamber) is not very significant, as when we take the
ratio between two ion chambers most of the factors that distinguish the
conceptual intensity from the measured signal will either cancel out, give
an arbitrary offset, or give a slowly varying monotonic drift with energy.
Thus, it is common to see experimental values reported for ``raw'' $\mu(E)$
in the literature that do not have dimensions of inverse length,   and which
might even have values that are negative.  For real values of $\mu(E)$ in
inverse length,   these measurements would be non-sensical, but for XAFS work
this is of no importance, as we'll subtract off a slowly varying background
anyway.

For fluorescence measurements,  the situation is similar, except that one uses
\begin{equation}
  \mu(E) = I_f/I
\end{equation}
\noindent where $I_f$ is the integrated fluorescence signal of interest.
As with the transmission measurements, there is generally no need to worry
about getting absolute intensities, and one can simply use the ratio of
measured intensities.    Because the instrumental drifts for a solid-state,
energy-discriminating fluorescence detector may be different than for a
gas-filled ion chamber, it is not unusual for $\mu(E)$ for fluorescence XAFS
measurements to have an overall upward drift with energy, where
transmission XAFS tends to drift down with energy.

In addition to the corrections for over-absorption and deadtime effects
discussed in the previous section, other corrections may need to be made to
the measured $\mu(E)$ data.  For example, sometimes bad glitches appear in
the data, and are not normalized away by dividing by $I_0$.  This is often
an indication of insufficient voltage on ion chambers, of too much harmonic
content in the X-ray beam, poorly uniform samples, incomplete deadtime
correction, or a combination of these.  If possible, it is preferred to
address these problems during the measurement, but this is not always
possible.  For such glitches, the best approach is simply to remove them
from the data -- asserting that they were not valid measurements of
$\mu(E)$.

Another example of a correction that can be made in the data reduction step
is for cases where another absorption edge occurs in the spectrum.  This
could be from the same element (as is over the case for measurements made
at the $L_{\rm III}$ edge, where the $L_{\rm II}$ edge will eventually be
excited, or from a different element in a complex sample.  As with a
glitch, the appearance of another edge means that $\mu(E)$ is no longer
from the edge and element of interest, and it is best to simply truncate
the data at the other edge.

\begin{Nfig}{width=3.5truein}{mu_deriv}
  \caption{The XANES portion of the XAFS spectrum (blue), and the
    identification of $E_0$ from the maximum of the derivative $d\mu/dE$
    (red).  This selection of $E_0$ is easily reproduced but somewhat
    arbitrary, so we may need to refine this value later in the analysis.}
  \label{Fig:RED:e0}
\end{Nfig}

\subsection{Pre-edge Subtraction and Normalization}

Once the measurement is converted to $\mu(E)$, the next step is usually to
identify the edge energy.  Since XANES features can easily move the edge by
several eV, and because calibrations vary between monochromators and
beamlines, it is helpful to be able to do this in an automated way that is
independent of the spectra.  Though clearly a crude approximation, the most
common approach is to take the maximum of the first derivative of $\mu(E)$.
Though it has little theoretical justification, it is easily reproduced,
and so can readily be checked and verified.


\begin{Nfig}{width=3.5truein}{mu_preedge}
  \caption{XAFS pre-edge subtraction and normalization. A line (or simple,
    low-order polynomial) is fit to the spectrum below the edge, and a
    separate low-order polynomial is fit to the spectrum well above the
    edge.  The edge jump, $\Delta\mu_0$, is approximated as the difference
    between these two curves at $E_0$.  Subtracting the pre-edge polynomial
    from the full spectrum and dividing by the edge jump gives a normalized
    spectrum.}
  \label{Fig:RED:preedge}
\end{Nfig}


Instrumental drifts from detector systems can be crudely approximated by a
simple linear dependence in energy.  That is a linear fit to the pre-edge
range of the measured spectrum is found, and subtracted.  In some cases, a
so-called Victoreen pre-edge function (in which one fits a line to $E^{n}
\mu(E)$ for some value of $n$, typically 1, 2 or 3).  Such a fit can do a
slightly better job at approximating the instrumental drifts for most XAFS
spectra, especially for dilute data measured in fluorescence with a
solid-state detector, where the contribution from elastic and
Compton-scattered intensity into the energy window of the peak of interest
will decrease substantially with energy, as the elastic peak moves up in
energy.

The next step in the process is to adjust the scale of $\mu(E)$ to account
for the absorption of 1 photo-electron.  By convention, we
{\emph{normalize}} the spectrum to go from approximately 0 below the edge
to approximately 1 above the edge.  To do this, we find the edge step,
$\Delta\mu$, and divide $\mu(E)$ by this value.   Typically, a low-order polynomial
is fitted to $\mu(E)$ well above the edge  (away from the XANES region),
and the value of this polynomial is extrapolated to $E_0$ to give the edge
step.  It should be emphasized that this convention is fairly crude and can
introduce systematic errors.


Examples of these processing steps (location of $E_0$, subtraction of
pre-edge, and normalization to an edge jump of 1) for transmission XAFS
data at the Fe $K$-edge of FeO are shown in Figures~\ref{Fig:RED:e0} and
~\ref{Fig:RED:preedge}.  For XANES analysis, this amount of data reduction
is generally all that is needed.  For both XANES and EXAFS analysis, the
most important part of these steps is the normalization to the edge step.
For XANES analysis, spectra are generally compared by amplitude, so an
error in the edge step for any spectra will directly affect the weight
given to that spectra.  For EXAFS, the edge step is used to scale
$\chi(k)$, and so is directly proportional to coordination number.  Errors
in the edge step will translate directly to errors in coordination number.
Getting good normalization (such that $\mu(E)$ goes to 1 above the edge) is
generally not hard, but requires some care, and it is important to assess
how well and how consistently this normalization process actually works for
a particular data set.  Most existing analysis packages do these steps
reasonably well, especially in making spectra be normalized consistently,
but it is not at all unusual for such automated, initial estimates of the
edge step to need an adjustment of 10\%.


\subsection{Background Subtraction}

\begin{Nfig}{width=3.5truein}{bkg}
  \caption{Post-edge background subtraction of FeO EXAFS. The background
    $\mu_0(E)$ shown in red is a smooth spline function that matches the
    low-$R$ components of $\mu(E)$, in this case using 1 {\AA} for $R_{\rm
      bkg}$.}
  \label{Fig:RED:bkg}
\end{Nfig}

Perhaps the most confusing and error-prone step in XAFS data reduction is
the determination and removal of the post-edge background function that
approximates $\mu_0((E)$. This is somewhat unfortunate, as it does not need
to be especially difficult.  Since $\mu_0(E)$ represents the absorption
coefficient from the absorbing atom without the presence of the neighboring
atoms, we cannot actually measure this function separately from the EXAFS.
In fact, even if possible, measuring $\mu(E)$ for an element in the gas
phases would not really be correct, as $\mu_0(E)$ represents the absorbing
atom embedded in the molecular or solid environment, just without the
scattering from the core electrons of the neighboring atoms.  Instead of
even trying to measure an idealized $\mu_0(E)$, we determine it empirically
by fitting a {\emph{spline}} function to $\mu(E)$.  A spline is a
piece-polynomial function that is designed to be adjustable enough to
smoothly approximate an arbitrary waveform, while maintaining convenient
mathematical properties such has continuous first and second derivatives.
This is certainly an {\emph{ad hoc}} approach, without any real physical
justification.  Still, it is widely used for EXAFS analysis, and has the
advantage of being able to account for those systematic drifts in our
measurement of $\mu(E)$ that make it differ from the true $\mu(E)$, as long
as those drifts vary slowly with energy.  The main challenge with using an
arbitrary mathematical spline to approximate $\mu_0(E)$ is to decide how
flexible to allow it to be, so as to ensure that it does not follow
$\mu(E)$ closely enough to remove the EXAFS.  That is, we want a $\mu_0(E)$
to remove the slowly varying parts of $\mu(E)$ while not changing
$\chi(k)$, the part of $\mu(E)$ that varies more quickly with $E$.


A simple approach for determining $\mu_0(E)$ that works well for most cases
relies on the Fourier transform to mathematically express the idea that
$\mu_0(E)$ should match the slowly varying parts of $\mu(E)$ while leaving
the more quickly varying parts of $\mu(E)$ to give the EXAFS $\chi$.  The
Fourier transform is critical to EXAFS analysis, and we'll discuss it in
more detail shortly, but for now the most important thing to know is that
it gives a weight for each frequency making up a waveform.  For EXAFS, the
Fourier transform converts $\chi$ from wavenumber $k$ to distance $R$.


\begin{Nfig}{width=3.5truein}{chik_kw}
  \caption{The EXAFS $\chi(k)$ (blue) isolated afer background subtraction.
    The EXAFS decays quickly with $k$, and weighting by $k^2$ (red)
    amplifies the oscillations at high $k$.}
  \label{Fig:RED:chik}
\end{Nfig}


For determining the background $\mu_0(E)$, we want a smoothly varying
spline function that removes the low-$R$ components of $\chi$, while
retaining the high-$R$ components.  Conveniently, we have a physically
meaningful measure of what distinguishes ``low-$R$'' from ``high-$R$'', in
that we can usually guess the distance to the nearest neighboring atom, and
therefore assert that there should be no signal in the EXAFS originating
from atoms at shorter $R$.  As a realistic rule of thumb, it is rare for
atoms to be closer together than about 1.5 {\AA} -- this is especially true
for the heavier elements for which EXAFS is usually applied.  Thus, we can
assert that a spline should be chosen for $\mu_0(E)$ that makes the
resulting $\chi$ have as little weight between 0 and 1 {\AA} as possible,
while ignoring the higher $R$ components of $\chi$.  This approach and the
use of $R_{\rm bkg}$ as the cutoff value for $R$\cite{autobk}, is not
always perfect, but can be applied easily to any spectra to give a spline
function that reasonably approximates $\mu_0(E)$ for most spectra with at
least some physically meaningful basis.  Figure~\ref{Fig:RED:bkg} shows a
typical background spline found for FeO, using a high-$R$ cutoff $R_{\rm
  bkg}$ of 1.0 {\AA}.  The resulting $\chi(k)$ is shown in
Figure~\ref{Fig:RED:chik}.

\begin{figure}[tbh]
\begin{center}
        \includegraphics*[width=3.5truein]{../figs/rbkg_mu}\vspace{0.25mm}

        \includegraphics*[width=2.75truein]{../figs/rbkg_chi}\hspace{-0.25mm}%%
        \includegraphics*[width=2.75truein]{../figs/rbkg_chir}
\end{center}
\caption{The effect of changing $R_{\rm bkg}$ on $\mu_0(E)$ and $\chi$.  A
  typical value for $R_{\rm bkg}$ of $1.0\,\text{\AA} $ (solid blue) results in
  a spline for $\mu_0(E)$ that can follow the low-$R$ variations in
  $\mu(E)$ while not removing the EXAFS.  A value too small ($R_{\rm bkg} =
  0.2\,\text{\AA} $, solid red) gives a spline that is not flexible enough,
  leaving a low-$R$ artifact, but one that will not greatly impact further
  analysis. On the other hand, a value too large ($R_{\rm bkg} =
  4.0\,\text{\AA} $, solid black) gives a spline flexible enough to completely
  remove the first and second shells of the EXAFS.}
  \label{Fig:RED:rbkg}
\end{figure}

The effect of varying $R_{\rm bkg}$ on the resulting spline for $\mu_0(E)$
and $\chi$ in both $k$- and $R$-space can be seen in
Figure~\ref{Fig:RED:rbkg}.  Here $\mu_0(E)$ for the same FeO $\mu(E)$
spectra is shown using values for $R_{\rm bkg}$ of 0.2, 1.0, and 4.0 {\AA}.
Having $R_{\rm bkg}$ too small (shown in red) results in a $\mu_0(E)$ that
is does not vary enough, giving a slow oscillation in $\chi(k)$, and
spurious peak below 0.5 {\AA} in $|\chi(R)|$.  On the other hand, setting
$R_{\rm bkg}$ too high (shown in black) can result in a $\mu_0(E)$ that
matches all the EXAFS oscillations of interest.  Indeed, with $R_{\rm bkg}
= 4\, \text{\AA}$, both the first and second shells of the FeO EXAFS are
entirely removed, leaving only the highest $R$ components.  This is clearly
undesirable.  In general, it is not too difficult to find a suitable value
for $R_{\rm bkg}$, with 1 {\AA} or half the near-neighbor distance being
fine default choices.  As we can see from Figure~\ref{Fig:RED:rbkg}, having
$R_{\rm bkg}$ too small is not always a significant problem -- the low $R$
peak can simply be ignored in the modeling of the spectra, and there is
little effect on the spectrum at higher $R$.

\subsection{EXAFS Fourier Transforms}

As mentioned above, the Fourier transform is central for the understanding
and modeling of EXAFS data.  Indeed, the initial understanding of the
phenomena was aided greatly by the ability to perform Fourier transforms on
measured EXAFS spectra.  While there are certainly ample resources
describing Fourier transforms, a few important points about the use of
Fourier transforms for EXAFS will be made here.

The first thing to notice from Figure~\ref{Fig:RED:chir} is that two peaks are
clearly visible -- these correspond to the Fe-O and Fe-Fe distances in
FeO.  Thus the Fourier transformed XAFS can be used to isolate and identify
different coordination spheres around the absorbing Fe atom.  Indeed,
$|\chi(R)|$ almost looks like a radial distribution function, $g(R)$.
While EXAFS does depend on the partial pair distribution -- the probability
of finding an atom at a distance $R$ from an atom of the absorbing species
-- $\chi(R)$ is certainly not just a pair distribution function.  This can
be seen from the additional parts to the EXAFS Equation, including the
non-smooth $k$ dependence of the scattering factor $f(k)$ and phase-shift
$\delta(k)$.

A very important thing to notice about $\chi(R)$ is that $R$ positions of
the peaks are shifted to lower $R$ from what $g(R)$ would give.  For FeO,
the first main peak occurs at 1.6 {\AA}, while the FeO distance in FeO is
more like 2.14 {\AA}. This is not an error, but is due to the scattering
phase-shift -- recall that the EXAFS goes as $\sin{[2kR +
  \delta(k)]}$. This phase-shift is typically -0.5 {\AA} or so, suggesting
that $\delta(k) \sim -k$, which can be verified as a decent approximation
from Figure~\ref{Fig:THE:scatt}.

\begin{Nfig}{width=3.5truein}{chir_magreal}
  \caption{The Fourier Transformed XAFS, $\chi(R)$. The magnitude
    $|\chi(R)|$ (blue) is the most common way to view the data, but the
    Fourier transform makes $\chi(R)$ a complex function, with both a real
    (red) and imaginary part, and the magnitude hides the important
    oscillations in the complex $\chi(R)$.}
  \label{Fig:RED:chir}
\end{Nfig}

The Fourier Transform results in a complex function for $\chi(R)$ even
though $\chi(k)$ is a strictly real function.  It is common to display only
the magnitude of $\chi(R)$ as shown on the left of Figure~\ref{Fig:RED:chir},
but the real and imaginary components contain important information that
cannot be ignored.  When we get to modeling the XAFS, it will be important
to keep in mind that $\chi(R)$ has both real and imaginary components.

The standard definition for a Fourier transform of a signal $f(t)$ can be written as

\begin{eqnarray}
  {\tilde{f}}(\omega) &=& \frac{1}{\sqrt{2\pi}} \int_{-\infty}^{\infty}
  f(t) e^{-i{\omega}t} dt \\
  f(t) &=& \frac{1}{\sqrt{2\pi}} \int_{-\infty}^{\infty}
  {\tilde{f}}(\omega) e^{i{\omega}t} d{\omega} \\
\end{eqnarray}
\noindent
where the symmetric normalization is one of the more common conventions.
This gives Fourier conjugate variables of $\omega$ and $t$, typically
representing frequency and time, respectively.

Because the XAFS equation (see Eq.~\ref{Eq:xafs1} for example) has $\chi(k)
\propto \sin[2kR + \delta(k)] $, the conjugate variables in XAFS are
generally taken to be $k$ and $2R$.  While the normalization for $\chi(R)$
and $\chi(k)$ is a matter of convention, we follow the symmetric case above
(with $t$ replaced by $k$ and $\omega$ replaced by $2R$).

There are a few important modifications to mention for the typical use of
Fourier transforms in XAFS analysis.  First, an XAFS Fourier transform
multiplies $\chi(k)$ by a power of $k$, typically $k^2$ or $k^3$, as shown
in Figure~\ref{Fig:RED:chik}.  This weighting helps compensate for the strong
decay with $k$ of $\chi(k)$, and allows either emphasizing different
portions of the spectra, or giving a fairly uniform intensity to the
oscillations over the $k$ range of the data.  In addition, $\chi(k)$ is
multiplied by a window function $\Omega(k)$ which acts to smooth the
resulting Fourier transform and remove ripple and ringing that would result
from a sudden truncation of $\chi(k)$ at the ends of the data range.

The second important issue is that the continuous Fourier transform
described above is replaced by a discrete transform.  This better matches
the discrete sampling of energy and $k$ values of the data, and allows Fast
Fourier Transform techniques to be used, which greatly improves
computational performance.  Using a discrete transform does change the
definitions of the transforms used somewhat. First, the $\chi(k)$ data must
be interpolated onto a {\emph{uniformly spaced}} set of $k$ values.
Typically, a spacing of $\delta k = 0.05\,\text{\AA}^{-1}$ is used.
Second, the array size for $\chi(k)$ used in the Fourier transform should
be a power of 2, or at least a product of powers of 2, 3, and 5.
Typically, $N_{\rm fft} = 2048$ points are used.  With the default spacing
between $k$ points, this would accommodate $\chi(k)$ up to $k = 102.4 \,
AA^{-1}$.  Of course, real experimental data doesn't extend that far, so
the array to be transformed is {\emph{zero-padded}} to the end of the
range.

The spacing of points in $R$ is given as $\delta R = \pi/{(N_{\rm fft}
  \delta k )}$.  The zero-padding of the extended $k$ range will increase
the density of points in $\chi(R)$ and result in smoothly interpolating the
values.  For $N_{\rm fft} = 2048$ and $\delta k = 0.05\, \text{\AA}^{-1}$, the
spacing in $R$ is approximately $\delta R = 0.0307 \,\text{\AA}$.

For the discrete Fourier transforms with samples of $\chi(k)$ at the points
$k_n = n \, \delta k$, and samples of $\chi(R)$ at the points $R_m = m \,
\delta R$, the definitions for the XAFS Fourier transforms become:

\begin{eqnarray*}
  \tilde\chi(R_m) &=& \frac{i \delta k}{\sqrt{\pi N_{\rm fft}}} \,
  \sum_{n=1}^{N_{\rm fft}} \chi(k_n) \,
  \Omega(k_n) \, k_n^w e^{2i\pi n m/N_{\rm fft}} \\
  \tilde\chi(k_n) &=& \frac{2 i \delta R}{\sqrt{\pi N_{\rm fft}}} \,
  \sum_{m=1}^{N_{\rm fft}} \tilde\chi(R_m) \,
  \Omega(R_m) \, e^{-2i\pi n m/N_{\rm fft}} \\
\end{eqnarray*}
\noindent
These normalization conventions preserve the symmetry properties of the Fourier
Transforms with conjugate variables $k$ and $2R$.

As mentioned above, the window function $\Omega(k)$ will smooth the
resulting Fourier transform and reduce the amount of ripple that would
arise from a sharp cut-off $\chi(k)$ at the ends of the data range.  Since
Fourier transforms are used widely in many fields of engineering and
science, there is an extensive literature on such window functions, and a
lot of choices and parameters available for constructing windows.  In
general terms, $\Omega(k)$ will gradually increases from 0 to 1 over the
low-$k$ region, and decrease form 1 to 0 over the high-$k$ region, and may
stay with a value 1 over some central portion.  Several functional forms
and parameters for these windoes can be used, and are available in most
EXAFS analysis software.  Many good examples of the shapes, parameters, and
effects of these on the resulting $\chi(R)$ are available in program
documentation, and other on-line tutorials.

In many analyses, the inverse Fourier transform is used to select a
particular $R$ range and transform this back to $k$ space, in effect
{\emph{filtering}} out most of the spectrum, and leaving only a narrow band
of $R$ values in the resulting filtered $\chi(k)$.  Such filtering has the
potential advantage of being able to isolate the EXAFS signal for a single
shell of physical atoms around the absorbing atom, and was how many of the
earliest EXAFS analyses were done.  This approach should be used with
caution since, for all but the simplest of systems, it can be surprisingly
difficult to effectively isolate the EXAFS contribution from an individual
scattering atom this way.  It is almost never possible to isolate a second
neighbor coordination sphere in this way.  For this reason, many modern
analyses of EXAFS will use a Fourier transform to convert $\chi(k)$ to
$\chi(R)$, and use $\chi(R)$ for data modeling, not bothering to try to use
a filter to isolate shells of atoms.
