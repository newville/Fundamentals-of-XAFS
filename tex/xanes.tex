\section{Interpretation of XANES}

Finally, we return to the XANES portion of the spectrum. Since XANES is a
much larger signal than EXAFS, XANES can be done at lower concentrations,
and at less-than-perfect sample conditions. The interpretation of XANES is
complicated by the fact that there is not a simple analytic (or even
physical) description of XANES. The main difficulty is that the EXAFS
equation breaks down at low-$k$, due to the 1/$k$ term and the increase in
the mean-free-path at very low-$k$. Still, there is much chemical
information from the XANES region, notably formal valence (very difficult
to experimentally determine in a nondestructive way) and coordination
environment. Figure~\ref{Fig:XANES:fe} shows the XANES spectra for several
iron compounds. Clearly, the edge position and shape is sensitive to formal
valence state, ligand type, and coordination environment.  If nothing else,
XANES can be used as a fingerprint to identify phases.

\begin{Nfig}{width=3.5truein}{fe_models_xanes}
  \caption{Fe $K$-edge XANES of Fe metal and several Fe compounds.}
  \label{Fig:XANES:fe}
\end{Nfig}

Even though there is not a useful ``XANES Equation'', we can go back to the
picture of section~\ref{Sect:theory}, and especially the concept of
available state for the photo-electron.  For $K$ shell absorption, where
the core-level is a $1s$ state, the photo-electron has to end up in a $p$
state (in general, the photo-electric effect changes the orbital quantum
number $l$ to $l\pm 1$).  Thus, even if there are available states with the
right energy, there might be no $1s$ absorption if there are no available
$p$ states.  For EXAFS, where the energies are well-above the threshold
energy, this is rarely an important concern.  For XANES, on the other hand,
this can play a very important role.  Transition metal oxides, for example,
usually have many unfilled $3d$ electrons near the Fermi level, and a
filled $3p$ band.  There are empty $2p$ electron states from the oxygen,
but these are too far away to appreciably over-lap with the metal $1s$
band.  Therefore, the metal $3d$ electrons do not normally participate in
the absorption process unless there is a strong hybridization of the O $2p$
and metal $3d$ levels.  The XANES spectra are then especially sensitive to
such hybridization.

The situation can be even more dramatic -- as Figure~\ref{Fig:XANES:cr} shows
for $\rm Cr^{3+}$ and $\rm Cr^{6+}$ oxides. For ions with unfilled
$d$-electrons bands, the pd hybridization is dramatically altered depending on
the coordination environment, which much stronger hybridization for
tetrahedral coordination than for octahedral coordination. Since the
photo-electron created due to a 1s core level (a $K$-shell) must have $p$-like
symmetry, the amount of overlap with the $d$-electron orbitals near the Fermi
level can dramatically alter the number of available states to the
$p$-electron, causing significant changes in the XANES spectrum. For the case
of Cr$^{\mathrm{6+}}$ pd hybridization results in a highly localized
molecular orbital state, giving a well-defined peak below the main
absorption edge, indicating a transition to a bound electronic state.

\begin{Nfig}{width=3.5truein}{cr_xanes}
  \caption{Cr $K$-edge XANES for $\rm Cr^{3+}$ and $\rm Cr^{6+}$
    oxides. Here the strong pre-edge peak in the $Cr^{6+}$ spectrum is a
    consequence of the tetrahedral symmetry causing considerable overdue of
    the empty $d$-electron orbitals with the $p$-states that the
    photo-electron must fill.}
  \label{Fig:XANES:cr}
\end{Nfig}


Though the lack of a simple analytic expression complicates XANES
interpretation, XANES can be described qualitatively (and nearly
quantitatively) in terms of

\begin{description}
\item[Coordination chemistry:] regular, distorted octahedral, tetrahedral
  coordination, as for Cr.

\item[Molecular orbitals:] $p-d$ orbital hybridization, crystal-field
  theory, and so on.

\item[Band-structure:] the density of available electronic states.

\item[Multiple-scattering:] multiple bounces of the photo-electron.
\end{description}

These chemical and physical interpretations are all related, of course.  As
discussed in the introduction, it all boils down to determining which
electronic states the photo-electron can fill.

An important and common application of XANES is to use the shift of the
edge position to determine the valence state. Figure~\ref{Fig:XANES:feo} shows
the valence dependence of Fe metal and oxides of $\rm Fe^{2+}$ and $\rm
Fe^{3+}$ (and a mixture of these two). With good model spectra, $\rm
Fe^{3+}/Fe^{2+}$ ratios can be determined with very good precision and
reliability. Similar ratios can be made for many other ions. The heights
and positions of pre-edge peaks can also be reliably used to empirically
determine oxidation states and coordination chemistry. These approaches of
assigning formal valence state based on edge features and as a
fingerprinting technique make XANES somewhat easier to crudely interpret
than EXAFS, even if a complete physical understanding of all spectral
features is not available.

For many systems, XANES analysis based on linear combinations of known
spectra from ``model compounds'' is sufficient to tell ratios
of valence states and/or phases. More sophisticated linear algebra
techniques such as Principle Component Analysis and Factor Analysis
can (and are) also be applied to XANES spectra.


\begin{Nfig}{width=3.5truein}{fe_oxides_xanes}
  \caption{Fe $K$-edge XANES of Fe metal and several Fe oxides, showing a clear
    relationship between edge position and formal valence state.  In addition,
    the shapes, positions, and intensities of pre-edge peaks can often be
    correlated to oxidation state.}
  \label{Fig:XANES:feo}
\end{Nfig}

XANES is considerably harder to fully interpret than EXAFS. Precise and
accurate calculations of all spectral features are still difficult,
time-consuming, and not always reliable. This situation is improving, but
at this point, quantitative analyses of XANES using {\emph{ab initio}}
calculations are very rare. Still, such calculations can help explain which
bonding orbitals and/or structural characteristics give rise to certain
spectral features, but are beyond the scope of this treatment, and will
have to wait for another day.

