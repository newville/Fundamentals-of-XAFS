\section{Introduction}

X-ray absorption fine structure (XAFS) describes the details of how X-rays
are absorbed by matter at energies near and above the binding energy of a
core electron level of a particular atom.  The X-ray absorption probability
near and above a core electron binding energy is altered by the chemical
and physical state of the atom.  XAFS spectra are especially sensitive to
the formal oxidation state, coordination chemistry, and the distances,
coordination number and species of the atoms immediately surrounding the
selected element.  Because of this, XAFS provides a practical and
relatively simple way to determine the chemical state and local atomic
structure for a selected atomic species, and is used routinely in a wide
range of scientific fields, including biology, environmental science,
catalysts research, and material science.  Since XAFS is fundamentally an
atomic probe, it places few constraints on the form of the samples that can
be studied, and can be used in a variety of systems and sample
environments.

All atoms have core level electrons, and XAFS can be measured for
essentially every element on the periodic table.  Importantly,
crystallinity is not required for XAFS, making it one of the few structural
probes available for noncrystalline and highly disordered materials,
including solutions.  Because X-rays are fairly penetrating in matter, XAFS
is not inherently surface-sensitive, though special measurement techniques
can be applied to enhance its surface sensitivity.  Because intense X-ray
sources can make very small beams, XAFS can be done on samples as small as
a few square microns.  In addition, many variations on experimental
techniques and sample conditions are available for XAFS, including
{\emph{in situ}} chemical processes and extreme conditions of temperature
and pressure.  XAFS measurements can be made on elements at trace abundance
in many systems, giving a unique and direct measurement of chemical and
physical state of dilute species in a variety of systems.

The X-ray absorption spectrum is typically divided into two regimes: X-ray
absorption near-edge spectroscopy (XANES) and extended X-ray absorption
fine-structure spectroscopy (EXAFS).  Though the two have the same physical
origin, this distinction is convenient for the interpretation. XANES is
strongly sensitive to formal oxidation state and coordination chemistry
(e.g., octahedral, tetrahedral coordination) of the absorbing atom, while
the EXAFS is used to determine the distances, coordination number, and
species of the neighbors of the absorbing atom.

XAFS measurements are relatively straightforward provided one has an
intense and energy-tunable source of X-rays.  In practice, this usually
means the use of synchrotron radiation to generate the X-rays.  Many
experimental stations at synchrotron sources around the world are specially
designed to enable XAFS measurements to be done fairly routinely.

The basic phenomena involving a beautiful mixture of modern physics and
chemistry is well-understood, an accurate theoretical treatment of XAFS is
fairly involved and still an area of active research, especially for the
XANES portion of the spectra.  A complete mastery of the data
interpretation can be somewhat challenging, though significant progress has
been made in analytical tools for XAFS in the past few decades.

XAFS is a mature technique, with a literature spanning many decades and
many disciplines.  As a result, several
books\cite{Teo1986,KoningsbergerPrins1988,GBunker2010,Calvin2013} have been
written specifically about XAFS, and one book on X-ray
physics\cite{AlsNielson2001} that covers XAFS.  There have been many
chapters and review articles written about XAFS, including early reviews of
the fledgling technique\cite{SternHeald1983}, complete theoretical
treatments\cite{RehrAlbersRMP2000}, and reviews focusing on applications in
a variety of fields, including mineralogy\cite{BrownRIMG1988} and soil
science\cite{Kelly2008}.  Earlier review articles
articles\cite{SuttonRIMG2002,ManceauRIMG2002} on synchrotron techniques in
geochemistry and environmental science also contain considerable
information on EXAFS.  In addition, several on-line
resources\cite{xafsorg,ixasportal} have lengthy tutorials and links to
software packages and documentation for XAFS.

Here,  the origins and interpretations of XAFS will be
introduced, with a hope of aiding the reader to be able to make
high-quality XAFS measurements as well as process and analyze the data.
The emphasis here is particularly on the processing and analysis of the
extended oscillations of the XAFS spectra, as the near-edge portion of the
spectra is covered in more detail elsewhere.  This chapter will not make
one an expert in XAFS, but but it should provide a firm foundation for a
new practitioner of XAFS.  The above citations are all strongly recommended
reading for further insights and different perspectives and emphasis.  The
reader is not expected to have previous experience with XAFS or X-ray
measurements, but some familiarity with advanced undergraduate-level
chemistry or physics and a knowledge of experimental practices and data
interpretation will be helpful. A longer, more detailed version of this
description is under review \cite{NewvilleRIMG2013}.

