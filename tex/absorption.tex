\section{X-Ray Absorption and Fluorescence}

X-rays are light with energies ranging from about 500 eV to 500 keV, or
wavelengths from about 25{\AA} to 0.25{\AA}.  In this energy regime, light
is absorbed by all matter through the \emph{photo-electric effect}, in
which an X-ray photon is absorbed by an electron in a tightly bound quantum
core level (such as the 1s or 2p level) of an atom, as shown in
Figure~\ref{Fig:XAS:photoelectric}.

In order for a particular electronic core level to absorb the X-ray, its
binding energy must be less than the energy of the incident X-ray.  If the
binding energy is greater than the energy of the X-ray, the bound electron
will not be perturbed from the well-defined quantum state and will not
absorb the X-ray.  If the binding energy of the electron is less than that
of the X-ray, the electron may be removed from its quantum level.  In this
case, the X-ray is destroyed (that is, absorbed) and any energy in excess
of the electronic binding energy is given to a photo-electron that is
ejected from the atom.  While this process has been well understood for a
century, the full implications of this process when applied to molecules,
liquids, and solids will give rise to XAFS.

\begin{Sfig}{width=3.0truein}{photoelectric}{xray_emission}
  \caption{Right: The photoelectric effect, in which an X-ray is absorbed
    by a atom and a core-level electron is promoted out of the atom,
    creating a photo-electron and leaving behind a hole in the core
    electron level.  Left: X-ray and Auger emission, in which the excited
    atomic after an absorption event will decay.  For either X-ray
    fluorescence or the Auger effect, an electron is moved from a less
    tightly bound orbital to the empty core level, and the energy
    difference between these levels is given to the emitted particle (X-ray
    or electron).  The emission energies have precise values that are
    characteristic for each atom, and can be used to identify the absorbing
    atom.  Since the probability of emission is directly proportional to
    the absorption probability, either X-ray fluorescence of Auger emission
    can be used to measure EXAFS and XANES.}
  \label{Fig:XAS:photoelectric}
\end{Sfig}

When discussing X-ray absorption, we are primarily concerned with the
\emph{absorption coefficient}, $\mu$ which gives the probability that
X-rays will be absorbed according to the Beer-Lambert Law:
\begin{equation}
  I=I_{0}e^{-{\mu}t}
\end{equation}
\noindent
where $I_{0}$ is the X-ray intensity incident on a sample, $t$ is the
sample thickness, and $I$ is the intensity transmitted through the sample,
as shown in Figure~\ref{Fig:XAS:BeersLaw}. For X-rays of sufficiently low
intensity, the X-ray intensity is proportional to the number of X-ray
photons.

\begin{Nfig}{width=2.0truein}{beer_lambert}
  \caption{X-ray absorption and the Beer-Lambert law:  An incident beam of
    monochromatic X-rays of intensity $I_0$ passes through a sample of
    thickness $t$, and the transmitted beam has intensity $I$.  The
    absorption coefficient $\mu$ is given by the Beer-Lambert
    law: $I = I_0 e^{-{\mu}t}$.}
  \label{Fig:XAS:BeersLaw}
\end{Nfig}

At most X-ray energies, the absorption coefficient $\mu $ is a smooth
function of energy, with a value that depends on the sample density $\rho$,
the atomic number Z, atomic mass $A$, and the X-ray energy $E$ roughly as
\begin{equation}
  \mu  \approx {\frac{{\rho}Z^4}{AE^3}}.
\end{equation}
\noindent
The strong dependence of $\mu$ on both $Z$ and $E$ is a fundamental
property of X-rays, and is the key to why X-ray absorption is useful for
medical and other imaging techniques such as X-ray computed tomography.
Figure~\ref{Fig:XAS:abscoeff} shows the energy-dependence of $\mu/\rho$ for O,
Fe, Cd, and Pb in the normal X-ray regime of 1 to 100 keV.  The values span
several orders of magnitude, so that good contrast between different
materials can be achieved for nearly any sample thickness and
concentrations by adjusting the X-ray energy.

\begin{Nfig}{width=3.5truein}{abscoeff}
  \caption{The absorption cross-section $\mu/\rho$ for several elements
    over the X-ray energy range of 1 to 100 keV.  Notice that there are at
    least 5 orders of magnitude in variation in $\mu/\rho$, and that in
    addition to the strong energy dependence, there are also sharp jumps in
    cross-section corresponding to the core-level binding energies of the
    atoms.}
  \label{Fig:XAS:abscoeff}
\end{Nfig}


When the incident X-ray has an energy equal to that of the binding energy
of a core-level electron, there is a sharp rise in absorption: an
\emph{absorption edge} corresponding to the promotion of the core level to
the continuum. For XAFS, we are concerned with the energy dependence of
$\mu$ at energies near and just above these absorption edges.  An XAFS
measurement is then simply a measure of the energy dependence of $\mu $ at
and above the binding energy of a known core level of a known atomic
species.  Since every atom has core-level electrons with well-defined
binding energies, we can select the element to probe by tuning the X-ray
energy to an appropriate absorption edge.  These absorption edge energies
are well-known (usually to within a tenth of percent), and tabulated.  The
edge energies vary with atomic number approximately as $Z^2$, and both $K$
and $L$ levels can be used in the hard X-ray regime (in addition, $M$ edges
can be for heavy elements in the soft X-ray regime), which allows most
elements to be probed by XAFS with X-ray energies between 4 and 35 keV.
Because the element of interest is
chosen in the experiment, XAFS is \emph{element-specific}.

Following an absorption event, the atom is said to be in an \emph{excited
<  state}, with one of the core electron levels left empty (a so-called
\emph{core hole}), and a \emph{photo-electron} emitted from the atom.  The
excited state will eventually decay (typically within a few femtoseconds)
of the absorption event. Though this decay does not affect the X-ray
absorption process, it is important for the discussion below.

There are two main mechanisms for the decay of the excited atomic state
following an X-ray absorption event, as shown in the right panel of
Figure~\ref{Fig:XAS:photoelectric}.  The first of these is X-ray
fluorescence , in which a higher energy electron core-level electron fills
the deeper core hole, ejecting an X-ray of well-defined energy.  The
fluorescence energies emitted in this way are characteristic of the atom,
and can be used to identify the atoms in a system, and to quantify their
concentrations. For example, an $L$ shell electron dropping into the $K$
level gives the $K_{\alpha }$ fluorescence line.

The second process for de-excitation of the core hole is the Auger Effect,
in which an electron drops from a higher electron level and a second
electron is emitted into the continuum (and possibly even out of the
sample).  In either case, a cascade of subsequent emissions will fill the
newly formed, less tightly bound hole until the atom is fully relaxed.
Either of these processes can be used to measure the absorption coefficient
$\mu$, though the use of fluorescence is somewhat more common.  In the hard
X-ray regime ($> 10\rm\,keV$), X-ray fluorescence is more likely to occur
than Auger emission, but for lower energy X-ray absorption, Auger processes
dominate.

XAFS can be measured by directly measure the intensity of X-rays
transmitted through a sample, shown in Figure~\ref{Fig:XAS:BeersLaw}, or by
monitoring a secondary emission such as x-ray fluorescence, or Auger
electrons, or in some cases even by monitoring visible light emitted by a
sample as part of the cascade of decay events.  We will return to the
details of the measurements later. For now it is enough to say that we can
measure the energy dependence of the X-ray absorption coefficient $\mu (E)$
either in transmission as
\begin{equation}
  \mu (E)=\ln (I_{0}/I)
\end{equation}
\noindent or in X-ray fluorescence (or Auger emission) as
\begin{equation}
  \mu(E) \propto I_{f}/I_{0}
\end{equation}
\noindent where $I_{f}$ is the monitored intensity of a fluorescence line
(or electron emission) associated with the absorption process.

A typical XAFS spectrum (measured in the transmission geometry for a powder
of FeO) is shown in Figure~\ref{Fig:XAS:mu}. The sharp rise in $\mu(E)$ due to
the Fe 1s electron level (near 7112 eV) is clearly visible in the spectrum,
as are the oscillations in $\mu(E)$ that continue well past the edge.  As
mentioned in the introduction, the XAFS is generally thought of in two
distinct portions: the near-edge spectra (XANES) -- typically within 30 eV
of the main absorption edge, and the extended fine-structure (EXAFS), which
can continue for a few keV past the edge.  As we shall, the basic physical
description of these two regimes is the same, but some important
approximations and limits allow us to interpret the extended portion of the
spectrum in a simpler and more quantitative way than is currently possible
for the near-edge spectra.

\begin{Nfig}{width=3.5truein}{mu_xanes_exafs}
  \caption{XAFS $\mu(E)$ for the Fe $K$ edge of FeO, showing
    the near-edge (XANES) region and
    the extended fine structure (EXAFS).}
  \label{Fig:XAS:mu}
\end{Nfig}

For the EXAFS, we are interested in the oscillations well above the
absorption edge, and define the EXAFS fine-structure function $\chi(E)$, as

\begin{eqnarray}
  \mu(E) &=& \mu_0(E) [ 1  + \chi(E)]  \\
  \chi(E) &=&   {\displaystyle  {\frac{{\mu(E) - \mu_0(E)}}{\Delta\mu}}}
  \label{Eq:mu2chi}
\end{eqnarray}
\noindent
where $\mu(E)$ is the measured absorption coefficient, $\mu_{0}(E)$ is a
smooth background function representing the absorption of an isolated atom,
and $\Delta \mu$ is the measured jump in the absorption $\mu(E)$ at
the threshold energy.

\begin{Sfig}{width=3.0truein}{xafs_chik}{xafs_chir}
  \caption{Isolated EXAFS for the Fe $K$ edge of FeO, shown weighted by
    $k^2$ (left) to emphasize the high-$k$ portion of the spectrum, and the
    Fourier transform of the $k$-weighted XAFS, $\chi(R)$ (right), showing
    the contribution from Fe-O and Fe-Fe neighbors.}
  \label{Fig:XAS:chi}
\end{Sfig}

As we will see, EXAFS is best understood in terms of the wave behavior of
the photo-electron created in the absorption process. Because of this, it
is common to convert the X-ray energy to $k$, the wave number of the
photo-electron, which has dimensions of 1/distance and is defined as

\begin{equation}
  k= \sqrt{ \frac{2m(E-E_0)}{{\hbar}^2}}
  \label{Eq:ABS:kdef}
\end{equation}
\noindent
where $E_0$ is the absorption edge energy, $m$ is the electron mass, and
$\hbar$ is Planck's constant.  The primary quantity for EXAFS is then
$\chi(k)$, the isolated variation in absorption coefficient as a function
of photo-electron wave number, and $\chi(k)$ is often referred to simply as
``the EXAFS''.  The EXAFS extracted from the Fe $K$-edge for FeO is shown
in the left panel of Figure~\ref{Fig:XAS:chi}.  The EXAFS signal is
oscillatory, and also decays quickly with $k$.  To emphasize the
oscillations, $\chi(k)$ is often multiplied by a power of $k$ typically
$k^{2}$ or $k^{3}$ for display, as is done for the plot in
Figure~\ref{Fig:XAS:chi}.

The different frequencies apparent in the oscillations in $\chi(k)$
correspond to different near-neighbor coordination shells.  This can be see
most clearly by applying a Fourier transform to the data, converting the
data from depending on wavenumber $k$ to depending on distance $R$.  As
seen in the right panel of Figure~\ref{Fig:XAS:chi}, the oscillations
present in the EXAFS $\chi(k)$) give rather well-defined peaks as a
function of $R$, corresponding to the distance from the absorbing atom to
its near neighbors.

A remarkable feature of EXAFS is that the contributions to the EXAFS from
scattering from different neighboring atoms can be described by a
relatively straightforward \emph{EXAFS Equation}, a simplified form of
which is
\begin{equation}
  \chi(k) = \sum_j {   \frac{N_j f_j(k) e^{-2k^2\sigma_j^2}}  {k{R_j}^2}
      \sin[2kR_j + \delta_j(k) ]}.
  \label{Eq:ABS:EXAFSEq}
\end{equation}
\noindent
Here $f(k)$ and $\delta(k)$ are scattering properties of the photo-electron
emitted in the absorption process by the atoms neighboring the excited
atom, $N$ is the number of neighboring atoms, $R$ is the distance to the
neighboring atom, and $\sigma^2$ is the disorder in the neighbor distance.
Though slightly complicated, the EXAFS equation is simple enough to enable
us to model EXAFS data reliably, in that we can determine $N$, $R$, and
$\sigma^2$ once we know the scattering amplitude $f(k)$ and phase-shifts
$\delta(k)$.  Furthermore, because these scattering factors depend on the
$Z$ of the neighboring atom, EXAFS is also sensitive to the atomic species
of the neighboring atom.
