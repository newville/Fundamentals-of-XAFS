\newcommand{\mytitle}{X-ray Absorption Fine-Structure Spectroscopy}
\newcommand{\myauthor}{Matthew Newville}

\begin{center}
  \Large{X-ray Absorption Fine-Structure Spectroscopy}
\end{center}
\vspace{5mm}

\begin{center} Matthew Newville \end{center}
\begin{center} Center for Advanced Radiation Sources\par
University of Chicago, Chicago, IL\par\end{center}

\vspace{5mm}

X-ray Absorption Fine-Structure (XAFS) is an element-specific spectroscopy
in which measurements are made by tuning the X-ray energy at and above a
selected core-level binding energy of a specified element.  Although XAFS
is a well-established technique providing reliable and useful information
about the chemical and physical environment of the probe atom, its
requirement for an energy-tunable X-ray source means it is primarily done
with synchrotron radiation sources and so is somewhat less common than
other spectroscopic analytical methods.  XAFS spectra are sensitive to the
oxidation state and coordination chemistry of the selected element, and the
extended oscillations in the spectra are sensitive to the distances,
coordination number and species of the atoms immediately surrounding the
selected element.  Because it is element-specific, XAFS places few
restrictions on the form of the sample, and can be used in a variety of
systems and bulk physical environments, including crystals, glasses,
liquids, and heterogeneous mixtures.  Additionally, XAFS can often be done
on low-concentration elements (typically down to a few ppm), and so has
applications in a wide range of scientific fields, including chemistry,
biology, catalysis research, material science, environmental science, and
geology.

\thispagestyle{empty}
\clearpage

