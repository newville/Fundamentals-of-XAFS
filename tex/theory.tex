\section{A Simple Theoretical Description of XAFS}{\label{Sect:theory}}

In this section, a simple physical description of the XAFS process and the
origin of the EXAFS Equation will be given.  Other useful treatments on a
similar level can be found in other
places\cite{Stern1988,RehrAlbersRMP2000} as well.  We start with the
photoelectric effect, now shown in Figure~\ref{Fig:THE:absorption}, in which
an X-ray of energy $E$ is absorbed by a core-level electron of a particular
atom with binding energy $E_0$.  Any energy from the X-ray in excess of
this binding energy is given to a photo-electron that propagates away from
the absorbing atom.  We will treat the photo-electron as a wave with
wavelength is proportional to $1/\sqrt{E-E_0}$.  It is most common to
describe the photo-electron by its {\emph{wavenumber}}, $k = 2\pi/\lambda$,
given in Eq.~\ref{Eq:ABS:kdef}.

\begin{Nfig}{width=3.5truein}{xafscartoon_bare}
  \caption{X-ray absorption by the photoelectric effect.  As the energy of
    the X-rays is increased above the energy of a core electron level,
    $E_0$, the probability of absorption has a sharp rise -- an edge jump.
    On absorbing the X-ray, the tightly bound core-level is destroyed, and
    a photo-electron with wavelength proportional to $1/\sqrt{(E-E_0)}$ is
    created.}
  \label{Fig:THE:absorption}
\end{Nfig}


The absorption of the X-ray by the particular core electron level requires
there to be an \emph{available quantum state} for the ejected
photo-electron to go to.  If no suitable state is available, there will be
no absorption from that core level.  At X-ray energies below the $1s$
binding energy (for example, below 7.1 keV for iron) the $1s$ electron
could only be promoted to a valence electron level below the Fermi level.
Since all the valence levels are filled, there is no state for the $1s$
electron to fill and so there is no absorption from that core-level.  Of
course, the sample isn't transparent to X-ray, as the higher level
electrons can be promoted into the continuum, but there is a sharp jump in
the probability of absorption as the X-ray energy is increased above a core
level binding energy.  These binding levels are often referred to as
{\emph{ absorption edges}} due to this strong increase in absorption
probability.

It should be noted that the quantum state that the photo-electron occupies
has not only the right energy, but also the right angular momentum.  For
photo-electric absorption, the angular momentum number must change by 1, so
that an $s$ core-level is excited into a $p$ state, while a $p$ core-level
can be excited into either an $s$ or $d$ level.  This is important for a
detailed, quantitative description of the XAFS, but is not crucial to basic
discussion of XAFS here, as we are generally dealing with energies far
above the continuum which have large density of states.  On the other hand,
the momentum state can be extremely important when considering XANES, as
the available energy states of the unfilled anti-bonding orbitals still
have well-defined and specific angular momentum states above the continuum
level.

\begin{Nfig}{width=3.5truein}{xafscartoon_scatter}
  \caption{The photo-electron can scatter from a neighboring atom and
    return to the absorbing atom.  This modulates the amplitude of the
    photo-electron wave-function at the absorbing atom, and also modulates
    the absorption coefficient $\mu(E)$.}
  \label{Fig:THE:exafs}
\end{Nfig}

When a neighboring atom is included in the picture, as shown in
Figure~\ref{Fig:THE:exafs}, the photo-electron can scatter from the
electrons of this atom, and some part of the scattered photo-electron can
return to the absorbing atom.  Of course, the simple one-dimensional
picture shown suggests that the probability of scattering the
photo-electron by the neighboring atom is quite large.  In fact, the
photo-electron wavefunction spreads radially out and has a fairly low
probability of scattering from the electrons in the neighboring atoms, so
that the EXAFS is a relatively small change in the total absorption
coefficient.

The key phenomenon for EXAFS is that some portion of the photo-electron
wavefunction is scattered from the neighboring atom, and returns to the
absorbing atom, all in a single coherent quantum state.  Since the
absorption coefficient depends on whether there is an available, unfilled
electronic state at the location of the atom and at the appropriate energy
(and momentum), the presence of the photo-electron scattered back from the
neighboring atom will alter the absorption coefficient:  This is the origin
of XAFS.

\subsection{The EXAFS equation}


\begin{Nfig}{width=3.5truein}{lambda}
  \caption{The photo-electron mean-free-path for XAFS $\lambda(k)$,
    representing how far the photo-electron can travel and still participate
    in the XAFS.  This term accounts for both the inelastic scattering of the
    photo-electron, and the finite lifetime of the core-hole.}
  \label{Fig:THE:lambda}
\end{Nfig}

We can arrive at the EXAFS equation (Eq.~\ref{Eq:ABS:EXAFSEq}) by
considering the photo-electron to be a spherical wavefunction leaving the
absorbing atom.  Recognizing that an electron in a solid or liquid can be
scattered by other electrons, we actually use a damped spherical
wavefunction,
\begin{equation}
  \psi(k,r) = {\frac{e^{ikr} e^{-r/\lambda(k)}}{kr}},
    \label{Eq:damped}
\end{equation}
\noindent
where, $\lambda$ is the {\emph{mean free path}} of the photo-electron,
representing how far it can typically travel before scattering
inelastically or before the core hole is filled by a decay process (which
typically happens within $10^{-15}$ s, or so)..  The mean-free-path is
typically 5 to 30 {\AA} and varies with $k$ with a fairly universal
dependence on $k$, as shown in Figure~\ref{Fig:THE:lambda}.

From the cartoon in Figure~\ref{Fig:THE:exafs}, we can see that the EXAFS
should be described as the photo-electron propagating away from the
absorbing, scattering from the neighbor atom a distance $R$ away, and
returning to the absorbing atom.  With the wavefunction for the
photo-electron above, we write this as

\begin{Sfig}{width=3truein}{scatt_amp}{scatt_pha}
  \caption{Photo-electron scattering amplitudes $f(k)$ (left) and
    phase-shifts $\delta(k)$ (right) for O, Fe, and Pb showing the
    dependence of these terms on atomic number Z. The variations in
    functional form allow Z to be determined ($\pm 5$ or so) from analysis
    of the EXAFS.}
  \label{Fig:THE:scatt}
\end{Sfig}

\begin{equation}
  \chi(k) \propto   {\frac{e^{ikR}e^{-R/\lambda(k)}}{kR}}\>
    [{2k f(k)e^{i\delta(k)}}] \>  {\frac{e^{ikR} e^{-R/\lambda(k)}}{kR}} + C.C.
    \label{Eq:xafs0}
\end{equation}
\noindent
where $f(k)$ and $\delta(k)$ are scattering properties of the neighboring
atom, and C.C. means complex conjugate.  As mentioned before, these
scattering factors depend on the Z of the neighboring atom, as illustrated
in Figure~\ref{Fig:THE:scatt} for a few elements.  Combining these terms in
and using the complex conjugate to make sure we end up with a real
function, we get

\begin{equation}
  \chi(k) =
    {\frac{f(k) e^{-2R/\lambda(k)}}{kR^2}}\>  \sin[2kR + \delta(k)]
\label{Eq:xafs1}
\end{equation}
\noindent
which looks much like the standard EXAFS equation.  So far, the treatment
describes one pair of absorbing atom and scattering atom, but for a real
measurement we'll average over billions of X-ray absorption events and so
atom pairs. Even for neighboring atoms of the same type, the thermal and
static disorder in the bond distances will give a range of distances that
will affect the XAFS. As a first approximation, the bonding environment and
disorder will change the XAFS equation from Eq.~\ref{Eq:xafs1} to
\begin{equation}
  \chi(k) =
      {{ \frac{N e^{-2k^2\sigma^2} f(k)e^{-2R/\lambda(k)} }{kR^2}}}\>
      \sin[2kR + \delta(k)]
      \label{Eq:xafs2}
\end{equation}
\noindent
where $N$ is the coordination number and $\sigma^2$ is the
mean-square-displacement in the bond distance $R$.

Of course, real systems usually have more that one type of neighboring atom
around a particular absorbing atom.  This is easily accommodated in the
XAFS formalism, as the measured XAFS will simply be a sum of the
contributions from each scattering atom type or \emph{coordination shell}:
\begin{equation}
  \chi(k) = \sum_j {
    {{ \frac{N_j e^{-2k^2\sigma_j^2}  e^{-2R/\lambda(k)} f_j(k)}{kR_j^2}}}\>
      \sin[2kR_j + \delta_j(k)] }
      \label{Eq:xafs3}
\end{equation}
\noindent
where $j$ represents the individual coordination shell of identical atoms
at approximately the same distance from the central atom.  In principle
there can be many such shells, but as shells of similar Z become close
enough (say, within a 0.05 {\AA} of each other), they become difficult to
distinguish from one another.

The use of $Ne^{-2k^2\sigma^2}$ above where $N$ is the coordination number
and $\sigma^2$ is the mean-square displacement of the set of interatomic
distances $R$ sampled by an EXAFS measurement is a simple way to account
for disorder in the set of near-neighbor distances in a system.  A real
EXAFS measurement samples billions of X-ray absorption events, and each of
these will sample a snapshot of the environment around a particular
absorbing atom.  For the present time, we'll model this distribution with a
two parameters: the average distance $R$ and the mean-square-displacement
$\sigma^2$.    More complete accounting of the range of distances seen would
use more complex models of distribution function of interatomic distances,
but we'll leave this aside for the present discussion.

Though we have a reasonably good explanation of the origin of the terms in
the EXAFS equation, there are a few topics to discuss before leaving this
section.  These are 1) polarization dependence, 2) intrinsic losses, and
3) multiple scattering.

X-ray emitted by synchrotron sources are highly polarized in the horizontal
plane, unless specifically altered by the source or X-ray optics.   The
emitted photo-electron preserves this polarization.  This can have
implications for EXAFS from ordered materials that do not have cubic
symmetry around the absorbing atom.

For $K$ shell edges, the dipole transition from the $s$ electron level means
the photo-electron goes as a $p$ wave, with electron probability having a
$\cos^2\theta$ dependence, with $\theta$ being the angle relative to the
horizontal plane.   Because of this, $K$ edge EXAFS is actually completely
blind to scattering atoms in the direction of the X-ray beam and  in the
vertical plane, above and below the absorbing atom.   For samples that are
isotropic, randomly oriented, or with cubic symmetry, the polarization is
of no consequence, but for anisotropic systems, and surfaces, this can
either be viewed as a complication, or exploited by carefully orienting the
sample relative to the horizontal plane.  Mathematically, we simply add
a $\epsilon \cdot \hat{R}$ term to the EXAFS equation which now becomes
\begin{equation}
  \chi(k) = \sum_j {
    {{ \frac{N_j e^{-2k^2\sigma_j^2}
          e^{-2R/\lambda(k)}(\epsilon \cdot \hat{R_j}) f_j(k) }{kR_j^2}}}\>
      \sin[2kR_j + \delta_j(k)] }
      \label{Eq:xafs_withpolar}
\end{equation}
\noindent
where $\epsilon$ is X-ray polarization vector, and $\hat{R}$ is the unit vector from
absorbing atom to scattering atom.  For  $L_{\rm{III}}$ and $L_{\rm{ II}}$
edges, as well as $M$ edges, the polarization dependence can be more
complex than the simple $\cos^2\theta$ dependence of $K$ shells due to the
fact that these edge have multiple allowed angular momentum states for the
outgoing photo-electron.

A second topic to discuss is the passive electron factor, $S_0^2$.  In the
description above, we ignored the relaxation due to the other electrons in
the excited atom.  That is, our ``initial state'' and ``final state'' above
should have been for the entire atom, but we considered only the core-level
electron. Writing ${| \Phi^{Z-1}_0 \rangle }$ for the remaining $Z-1$
electrons in unexcited atom, and ${\langle \Phi^{Z-1}_f|}$ for the $Z-1$
electrons in the excited atoms, we need to place a factor $S_0^2 = {
  |{\langle \Phi^{Z-1}_f |\Phi^{Z-1}_0 \rangle}|^2} $ as a pre-factor to
the EXAFS equation.  Though recent research has suggested that $S_0^2$ may
have some $k$ dependence, especially at low $k$, it is usually interpreted
simply as a constant value, so that the EXAFS equation becomes
\begin{equation}
  \chi(k) = \sum_j {
    {{\frac{S_0^2 N_j e^{-2k^2\sigma_j^2} e^{-2R_j/\lambda(k)}  (\epsilon \cdot \hat{R_j}) f_j(k)}{kR_j^2}}}\>
    \sin[2kR_j + \delta_j(k)] }.
  \label{Eq:xafs_withs02}
\end{equation}
\noindent
Though $S_0^2$ is assumed to be constant, and generally found to be $ 0.7 < S_0^2
< 1.0 $.  By far the biggest consequence of this is that this factor is
{\emph{completely correlated with $N$}} in the EXAFS equation.  This fact,
along with the data reduction complication discussed later that the edge
step $\Delta\mu$ in Eq.~\ref{Eq:mu2chi} is challenging to determine
experimentally, makes absolute values for the coordination number $N$
difficult to determine with high accuracy.


The third and final topic to discuss is multiple scattering. So far the
treatment of EXAFS has implied that the photo-electron always scatters from
one neighboring atom and returns to the absorber.  In fact, the
photo-electron can scatter from more than one neighboring atom, making a
more convoluted {\emph{scattering path}} than simply to one scattering atom
and back.  Examples of the more important types of multiple scattering
paths are illustrated in Figure~\ref{Fig:THE:ms}.

\begin{Nfig}{width=3.0truein}{mspaths}
  \caption{Multiple scattering paths for the photo-electron.  While
    single-scattering paths generally dominate most EXAFS spectra, multiple
    scattering paths can give important contributions, especially in
    well-ordered crystalline materials.  Fortunately, these terms can be
    included into the standard EXAFS formalism.}
  \label{Fig:THE:ms}
\end{Nfig}


Multiple scattering paths can give important contributions for EXAFS,
especially beyond the first coordination shell, and are nearly always
important for XANES.  In general, most first-shell analysis of EXAFS is not
strongly affected by multiple scattering, but second-shell analysis can be,
and shells beyond the second are almost always complicated by
multiple-scattering paths.  For highly-ordered crystalline materials,
focused linear multiple scattering paths, as shown Figure~\ref{Fig:THE:ms} can
be particularly important, and neglecting them in an analysis can give
erroneous results.

Though the details of the calculations are beyond the scope of this work
\cite{RehrAlbersRMP2000}, accounting for multiple scattering formally in
the EXAFS equation is conceptually quite easy.  We can simply change the
meaning of the sum in Eq.~\ref{Eq:xafs3} to be a sum over {\emph{scattering
    paths}}, including multiple scattering path, instead of being a sum
over coordination shells.  We also have to change our interpretation of $R$
from ``interatomic distance'' to ``half path length''.  In addition, our
scattering amplitudes $f(k)$ and phase-shifts $\delta(k)$ now need to
include the contribution from each scattering atom in the path, so that the
term in the EXAFS equation can be said to be {\emph{effective}} scattering
amplitudes and phase-shifts.  Unfortunately, the existence of multiple
scattering means that the number of paths needed to properly account for an
EXAFS spectra grows quickly (exponentially) with path distance.  This puts
a practical limit on our ability to fully interpret EXAFS spectra from
completely unknown systems.

We've used a simple physical picture of photo-electron scattering to
develop the EXAFS equation that we can use in the quantitative analysis of
EXAFS spectra,
\begin{equation}
  \chi(k) = \sum_j {
    {{\frac{S_0^2 N_j e^{-2k^2\sigma_j^2} e^{-2R_j/\lambda(k)}
 (\epsilon \cdot \hat{R_j}) f_j(k)}{kR_j^2}}}\>
    \sin[2kR_j + \delta_j(k)] }.
  \label{Eq:xafs_final}
\end{equation}
\noindent
From this, our final version of the EXAFS equation, we can draw a
few physical conclusions about XAFS.  First, because of the $\lambda(k)$
term and the $R^{-2}$ term, XAFS is seen to be an inherently \emph{local
  probe}, not able to see much further than 5 {\AA} or so from the
absorbing atom.   Second, the XAFS oscillations consist of different
frequencies that correspond to the different distances of atomic shells.
This will lead us to use Fourier transforms in the analysis.  Finally, in
order to extract the distances and coordination numbers, we need to have
accurate values for the scattering amplitude and phase-shifts $f(k)$ and
$\delta(k)$ from the neighboring atoms.


