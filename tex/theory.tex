\section{A Simple Theoretical Description of XAFS}{\label{Sect:theory}}

In this section, a simple physical description of the XAFS process and the
origin of the EXAFS Equation will be given.  Other useful treatments on a
similar level can be found in other
places\cite{Stern1988,RehrAlbersRMP2000} as well.  We start with the
photoelectric effect, now shown in Figure~\ref{Fig:THE:absorption}, in
which an X-ray of energy $E$ is absorbed by a core-level electron of a
particular atom with binding energy $E_0$.  Any energy from the X-ray in
excess of this binding energy is given to a photo-electron that propagates
away from the absorbing atom.  We will treat the photo-electron as a wave,
noting that its wavelength is proportional to $1/\sqrt{E-E_0}$.  It is most
common to describe the photo-electron by its {\emph{wavenumber}}, $k =
2\pi/\lambda$, given in Eq.~\ref{Eq:ABS:kdef}.

\begin{Nfig}{width=3.5truein}{xafscartoon_bare}
  \caption{Cartoon of X-ray absorption by the photoelectric effect.  As the
    energy of the X-rays is increased to just above the energy of a tightly
    bound core electron level, $E_0$, the probability of absorption has a
    sharp rise -- an edge jump.  In the absorption process, the tightly
    bound core-level is destroyed, and a photo-electron is created. The
    photo-electron travels as a wave with wavelength proportional to
    $1/\sqrt{(E-E_0)}$.}
  \label{Fig:THE:absorption}
\end{Nfig}


The absorption of the X-ray by the particular core electron level requires
there to be an \emph{available quantum state} for the ejected
photo-electron to go to.  If no suitable state is available, there will be
no absorption from that core level.  At X-ray energies below the $1s$
binding energy (for example, below 7.1 keV for iron) the $1s$ electron
could only be promoted to a valence electron level below the Fermi level --
there is simply not enough energy to put the electron into the conduction
band.  Since all the valence levels are filled, there is no state for the
$1s$ electron to fill, and so there is no absorption from that core-level.
Of course, a sample is not transparent to X-rays with energies below the
$1s$ binding level, as the higher level electrons can be promoted into the
continuum, but there is a sharp jump in the probability of absorption as
the X-ray energy is increased above a core level binding energy.  These
binding levels are often referred to as {\emph{ absorption edges}} due to
this strong increase in absorption probability.

It should be noted that the quantum state that the photo-electron occupies
has not only the right energy, but also the right angular momentum.  For
photo-electric absorption, the angular momentum number must change by 1, so
that an $s$ core-level is excited into a $p$ state, while a $p$ core-level
can be excited into either an $s$ or $d$ level.  This is important for a
detailed, quantitative description of the XAFS, but is not crucial to basic
discussion of XAFS here, as we are generally dealing with energies far
above the continuum which have large density of states.  On the other hand,
the momentum state can be extremely important when considering XANES, the
near-edge portion of the spectra, as the available energy states of the
unfilled anti-bonding orbitals still have well-defined and specific angular
momentum states above the continuum level.

\begin{Nfig}{width=3.5truein}{xafscartoon_scatter}
  \caption{The photo-electron can scatter from a neighboring atom and
    return to the absorbing atom.  This modulates the amplitude of the
    photo-electron wave-function at the absorbing atom, and also modulates
    the absorption coefficient $\mu(E)$, causing the XAFS.}
  \label{Fig:THE:exafs}
\end{Nfig}

The picture above described absorption for an isolated atom.  When a
neighboring atom is included in the picture, as shown in
Figure~\ref{Fig:THE:exafs}, the photo-electron can scatter from the
electrons of this neighboring atom, and some part of the scattered
photo-electron can return to the absorbing atom.  Of course, the simple
one-dimensional picture shown suggests that the probability of scattering
the photo-electron by the neighboring atom is quite large.  In
a real, three dimensional sample, the photo-electron wavefunction spreads radially out and has a fairly low
probability of scattering from the electrons in the neighboring atoms, so
that the EXAFS is a relatively small change in the total absorption
coefficient.

The key phenomenon for EXAFS is that some portion of the photo-electron
wavefunction is scattered from the neighboring atom, and returns to the
absorbing atom, all in a single coherent quantum state.  Since the
absorption coefficient depends on whether there is an available, unfilled
electronic state at the location of the atom and at the appropriate energy
(and momentum), the presence of the photo-electron scattered back from the
neighboring atom will alter the absorption coefficient:  This is the origin
of XAFS.

\subsection{The EXAFS equation}

We'll now spend some effort developing the standard EXAFS equation using a
slightly more formal description of this simple physical picture above, but
still somewhat less rigorous than a full-blown quantum mechanical
description.  The goal here is to describe enough of the basic physics to
identify where the different components of the EXAFS equation arise from,
and so what they mean for use in the analysis of spectra.

Since X-ray absorption is a {\emph{transition}} between two quantum states
(from an initial state with an X-ray, a core electron, and no
photo-electron to a final state with no X-ray, a core hole, and a
photo-electron), we describe $\mu(E)$ with Fermi's Golden Rule:

\begin{equation}
 \mu(E) \propto   { | \langle i | {\cal{H}} | f \rangle |^2   }
 \label{Eq:fermi}
\end{equation}
\noindent
where $ \langle i | $ represents the initial state (an X-ray, a core
electron, and no photo-electron), $|f\rangle$ is the final state (no X-ray,
a core-hole, and a photo-electron), and $\cal{H}$ is the interaction term,
which we'll come back to shortly.  Since the core-level electron is very
tightly bound to the absorbing atom, the initial state will not be altered
by the presence of the neighboring atom, at least to first approximation.
The final state, on the other hand, will be affected by the neighboring
atom because the photo-electron will be able to scatter from it.  If we
expand $|f\rangle$ into two pieces, one that is the ``bare atom'' portion
($|f_0\rangle$), and one that is the effect of the neighboring atom
($|\Delta f\rangle$) as
\begin{equation}
  |f\rangle = |f_0\rangle + |\Delta f\rangle,
\end{equation}
\noindent
we can then expand Eq.~\ref{Eq:fermi} to
\begin{equation}
  \mu(E) \propto {{| \langle i | {\cal{H}} | f_0 \rangle |^2}}
  \bigl[ 1 + {{\langle i | {\cal{H}}| \Delta f \rangle }}
    { \frac{   {\langle f_0 | {\cal{H}} | i \rangle }^{*}
      }{ | \langle i | {\cal{H}} | f_0 \rangle |^2 } }  + C.C \bigr]
\end{equation}
\noindent
where C.C. means complex conjugate. We've arranged the terms here so that
this expression resembles a slight variation on our previous relationship
between $\mu(E)$ and
$\chi(E)$   in Eq.~\ref{Eq:mu2chi},
\begin{equation}
  \mu(E)  = \mu_0(E) [ 1 + \chi(E)].
\end{equation}
\noindent
where we're allowing the $\Delta\mu_0$ in Eq.~\ref{Eq:mu2chi} to be the
energy-dependent $\mu_0(E)$ .  We can now assign $\mu_0 = | \langle i |
{\cal{H}} | f_0 \rangle |^2 $ as the ``bare atom absorption'', which
depends only on the absorbing atom -- as if the neighboring atom wasn't
even there.  We can also see that the fine-structure $\chi$ will be
proportional to the term with $|\Delta f\rangle$:
\begin{equation}
  \chi(E)  \propto \langle i | {\cal{H}}| \Delta f \rangle.
\end{equation}
\noindent
which indicates that the EXAFS is due to the interaction of the scattered
portion of the photo-electron and the initial absorbing atom.

We can work this term out as an integral equation fairly easily, if
approximately.  The interaction term ${\cal{H}}$ represents the process of
changing between two energy, momentum states.  In quantum radiation theory,
the interaction term needed is the ${p{\cdot}A}$ term, where ${A}$ is the
quantized vector potential (there is also an ${A{\cdot}A}$ term, but this
does not contribute to absorption).  For the purposes here, this reduces to
a term that is proportional to $e^{ikr}$.  The initial state is a tightly
bound core-level, which we can approximate by a delta function (a 1s level
for atomic number $Z$ extends to around $a_0 / Z$, where $a_0$ is the Bohr
radius of $\approx 0.529\,\text{\AA}$, so this is a good approximation for
heavy elements, but less good for very light elements).  The change in
final state is just the wave-function of the scattered photo-electron,
$\psi_{\rm{scatter}}(r)$.  Putting these terms together gives a simple
expression for the EXAFS:

\begin{Sfig}{width=3truein}{scatt_amp}{scatt_pha}
  \caption{Functional forms for $f(k)$ (left) and $\delta(k)$ (right) for
    O, Fe, and Pb showing the dependence of these terms on atomic number
    Z. The variations in functional form allow Z to be determined ($\pm 5$
    or so) from analysis of the EXAFS.}
  \label{Fig:THE:scatt}
\end{Sfig}

\begin{equation}
\chi(E) \propto \int dr \delta(r) e^{ikr}
      \psi_{\rm{scatter}}(r)  =  \psi_{\rm{scatter}}(0).
\end{equation}
\noindent
In words, this simply states the physical picture shown in
Figure~\ref{Fig:THE:exafs}:

\begin{center}
  \fcolorbox{black}{white}{
    \begin{minipage}{4.0truein}
      \textbf{The EXAFS $\chi(E)$ is proportional to the amplitude of the
        scattered photo-electron at the absorbing atom.}
  \end{minipage}
  }
\end{center}


We can now evaluate the amplitude of the scattered photo-electron at the
absorbing atom to get the EXAFS equation.  Using the simple physical
picture from Figure~\ref{Fig:THE:exafs}, we can describe the outgoing
photo-electron wave-function $\psi(k,r)$ traveling as a spherical wave,
\begin{equation}
  \psi(k,r) = {\frac{e^{ikr}}{kr}},
  \label{Eq:spherical}
\end{equation}

\noindent
traveling a distance $R$ to the neighboring atom, then scattering from a
neighbor atom, and traveling as a spherical wave a distance $R$ back to the
absorbing atom. We simply multiply all these factors together to get
\begin{equation}
  \chi(k) \propto \psi_{\rm{scatter}}(k,r=0) =
  {\frac{e^{ikR}}{kR}}\>
    [{2k f(k)e^{i\delta(k)}}] \>
    {\frac{e^{ikR}}{kR}} + C.C.
    \label{Eq:xafs0}
\end{equation}
\noindent
where $f(k)$ and $\delta(k)$ are scattering properties of the neighboring
atom, and C.C. means complex conjugate.  As mentioned before, these
scattering factors depend on the Z of the neighboring atom, as illustrated
in Figure~\ref{Fig:THE:scatt} for a few elements.  Combining these terms in
and using the complex conjugate to make sure we end up with a real
function, we get

\begin{equation}
  \chi(k) =
    {\frac{f(k)}{kR^2}}\>
    \sin[2kR + \delta(k)]
\label{Eq:xafs1}
\end{equation}
\noindent
which looks much like the standard EXAFS equation.  For
mathematical convenience,  the EXAFS Equation is sometimes written with the
$\sin$ term replaced with the imaginary part of an exponential:
\begin{equation}
  \chi(k) =
    {\frac{f(k)}{kR^2}}\> {\rm{Im}}  [e^{i[2kR + \delta(k)]}]
\label{Eq:xafs1a}
\end{equation}
we'll use this form on occasion.

The treatment to get to Eq.~\ref{Eq:xafs1} was for one pair of absorbing
atom and scattering atom, but for a real measurement we'll average over
billions of X-ray absorption events and so atom pairs. Even for neighboring
atoms of the same type, the thermal and static disorder in the bond
distances will give a range of distances that will affect the XAFS. As a
first approximation, the bonding environment and disorder will change the
XAFS equation from Eq.~\ref{Eq:xafs1} to
\begin{equation}
  \chi(k) =
      {{ \frac{N e^{-2k^2\sigma^2} f(k)}{kR^2}}}\>
      \sin[2kR + \delta(k)]
      \label{Eq:xafs2}
\end{equation}
\noindent
where $N$ is the coordination number and $\sigma^2$ is the
mean-square-displacement in the bond distance $R$.    We'll return to this
topic later.

Of course, real systems usually have more that one type of neighboring atom
around a particular absorbing atom.  This is easily accommodated in the
XAFS formalism, as the measured XAFS will simply be a sum of the
contributions from each scattering atom type or \emph{coordination shell}:
\begin{equation}
  \chi(k) = \sum_j {
    {{ \frac{N_j e^{-2k^2\sigma_j^2} f_j(k)}{kR_j^2}}}\>
      \sin[2kR_j + \delta_j(k)] }
      \label{Eq:xafs3}
\end{equation}
\noindent
where $j$ represents the individual coordination shell of identical atoms
at approximately the same distance from the central atom.  In principle
there can be many such shells, but as shells of similar Z become close
enough (say, within a 0.05 {\AA} of each other), they become difficult to
distinguish from one another.

The explanation so far of what goes into the EXAFS equation gives the most
salient features of the physical picture for EXAFS.  but ignores many
nuances.  In order to be able to quantitatively analyze EXAFS in real
systems, we'll need to cover some of these subtleties, giving four main
points to discuss.  These are 1) the finite photo-electron mean-free-path,
2) the relaxation due to the passive (non-core) electrons of the excited
atom, 3) multiple-scattering of the photo-electron,  4) a more detailed
treatment of structural and thermal disorder, and 5) a brief description of
the effect of X-ray polarization.

\subsection{$\lambda(k)$: The inelastic mean-free-path}

\begin{Nfig}{width=3.0truein}{lambda}
  \caption{The photo-electron mean-free-path for XAFS $\lambda(k)$,
    representing how far the photo-electron can travel and still participate
    in the XAFS.  This term accounts for both the inelastic scattering of the
    photo-electron, and the finite lifetime of the core-hole.}
  \label{Fig:THE:lambda}
\end{Nfig}

The most significant approximation we made above was to assert that the
outgoing photo-electron went out as a spherical wave, as given in
Eq.~\ref{Eq:spherical}.  In doing so, we neglected the fact that the
photo-electron can also scatter {\emph{inelastically}} from other sources
-- other conduction electrons, phonons, and so on.  In order to participate
in the XAFS, the photo-electron has to scatter from the neighboring atom
and return to the absorbing atom {\emph{elastically}} (that is, at the same
energy) as the outgoing photo-electron.  In addition, the scattered portion
of the photo-electron has to make it back to the absorbing atom before the
excited state decays (that is, before the core-hole is filled through the
Auger or fluorescence process).  To account for both the inelastic
scattering and the finite {\emph{core-hole lifetime}}, we can use a damped
spherical wave:
\begin{equation}
  \psi(k,r) = {\frac{e^{ikr} e^{-r/\lambda(k)}}{kr}},
    \label{Eq:damped}
\end{equation}
\noindent
for the photo-electron wave-function in place of the spherical wave of
Eq.~\ref{Eq:spherical}.  Here, $\lambda$ is the {\emph{mean free path}} of
the photo-electron, representing how far it can typically travel before
scattering inelastically or before the core-hole is filled.  The core-hole
lifetime is on the order of $10^{-15}$ s, depending somewhat on the energy
of the core-level.  The mean-free-path is typically 5 to 30 {\AA} and
varies with $k$ with a fairly universal dependence on $k$, shown in
Figure~\ref{Fig:THE:lambda}.  Including this $\lambda(k)$, the EXAFS equation
becomes
\begin{equation}
  \chi(k) = \sum_j {
    {{ \frac{N_j e^{-2k^2\sigma_j^2} e^{-2R_j/\lambda(k)}  f_j(k)}{kR_j^2}}}\>
    \sin[2kR_j + \delta_j(k)] }
  \label{Eq:xafs4}
\end{equation}
\noindent
It is the finite size of $\lambda$, as well as the $1/R^2$ term (which also
originates from the wavefunction of the outgoing photo-electron) in the
EXAFS equation that shows EXAFS to be a local probe, insensitive to atomic
structure beyond 10 {\AA} or so.

As an aside, we note that it is possible to treat the losses that are
described by $\lambda(k)$ as a complex wavenumber, so that $k$ becomes $p =
k + i/\lambda$, and the EXAFS Equation can be written with $p$ instead of
$k$.  This reflects the common usage in the theoretical condensed matter
physics literature that the photo-electron energy is complex, and so
includes the effects of the mean-free-path not only in a $e^{-2R/\lambda}$
term, but also in the disorder terms, which can be important in some
analyses.  This can be incorporated into quantitative analysis tools, but
is beyond the scope of the present work, so we will continue to use the
form of the EXAFS Equation above, with the explicit $\lambda$ term.

\subsection{$S_0^2$ : intrinsic losses}

A second approximation we made in the description above was to ignore the
relaxation due to the other electrons in the excited atom.  That is, our
``initial state'' and ``final state'' above should have been for the entire
atom, but we considered only the core-level electron. Writing ${|
  \Phi^{Z-1}_0 \rangle }$ for the remaining $Z-1$ electrons in unexcited
atom, and ${\langle \Phi^{Z-1}_f|}$ for the $Z-1$ electrons in the excited
atoms, we end up with a factor of
\begin{equation}
  S_0^2 =  {  |{\langle \Phi^{Z-1}_f |\Phi^{Z-1}_0 \rangle}|^2}
  \label{Eq:S02}
\end{equation}
\noindent
that can be placed in front of the EXAFS equation.  Though recent research
has suggested that $S_0^2$ may have some $k$ dependence, especially at low
$k$, it is usually interpreted simply as a constant value, so that the
EXAFS equation becomes
\begin{equation}
  \chi(k) = \sum_j {
    {{\frac{S_0^2 N_j e^{-2k^2\sigma_j^2} e^{-2R_j/\lambda(k)}  f_j(k)}{kR_j^2}}}\>
    \sin[2kR_j + \delta_j(k)] }
  \label{Eq:xafs_withs02}
\end{equation}
\noindent
which is the final form of the EXAFS equation that we will use for
analysis.


$S_0^2$ is assumed to be constant, and generally found to be $ 0.7 < S_0^2
< 1.0 $.  By far the biggest consequence of this is that this factor is
{\emph{completely correlated with $N$}} in the EXAFS equation.  This fact,
along with the data reduction complication discussed later that the edge
step $\Delta\mu$ in Eq.~\ref{Eq:mu2chi} is challenging to determine
experimentally, makes absolute values for the coordination number $N$
difficult to determine with high accuracy.

\subsection{Multiple scattering of the photo-electron}

\begin{Nfig}{width=3.0truein}{mspaths}
  \caption{Multiple scattering paths for the photo-electron.  While
    single-scattering paths generally dominate most EXAFS spectra, multiple
    scattering paths can give important contributions, especially in
    well-ordered crystalline materials.  Fortunately, these terms can be
    included into the standard EXAFS formalism.}
  \label{Fig:THE:ms}
\end{Nfig}

So far the treatment of EXAFS has implied that the photo-electron always
scatters from one neighboring atom and returns to the absorber.  In fact,
the photo-electron can scatter from more than one neighboring atom, making
a more convoluted {\emph{scattering path}} than simply to one scattering
atom and back.  Examples of the more important types of multiple scattering
paths are illustrated in Figure~\ref{Fig:THE:ms}.

Multiple scattering paths can give important contributions for EXAFS,
especially beyond the first coordination shell, and are nearly always
important for XANES.  In general, most first-shell analysis of EXAFS is not
strongly affected by multiple scattering, but second-shell analysis can be,
and shells beyond the second are almost always complicated by
multiple-scattering paths.  For highly-ordered crystalline materials,
focused linear multiple scattering paths, as shown Figure~\ref{Fig:THE:ms} can
be particularly important, and neglecting them in an analysis can give
erroneous results.

Though the details of the calculations are beyond the scope of this work
\cite{RehrAlbersRMP2000}, accounting for multiple scattering formally in
the EXAFS equation is conceptually quite easy.  We can simply change the
meaning of the sum in Eq.~\ref{Eq:xafs3} to be a sum over {\emph{scattering
    paths}}, including multiple scattering path, instead of being a sum
over coordination shells.  We also have to change our interpretation of $R$
from ``interatomic distance'' to ``half path length''.  In addition, our
scattering amplitudes $f(k)$ and phase-shifts $\delta(k)$ now need to
include the contribution from each scattering atom in the path, so that the
term in the EXAFS equation can be said to be {\emph{effective}} scattering
amplitudes and phase-shifts.  Unfortunately, the existence of multiple
scattering means that the number of paths needed to properly account for an
EXAFS spectra grows quickly (exponentially) with path distance.  This puts
a practical limit on our ability to fully interpret EXAFS spectra from
completely unknown systems.

\subsection{Disorder terms and $g(R)$ }

We gave a simple description of disorder above, using $Ne^{-2k^2\sigma^2}$
in the EXAFS equation, where $N$ is the coordination number and $\sigma^2$
is the mean-square displacement of the set of interatomic distances $R$
sampled by an EXAFS measurement.  As noted above, the core-hole lifetime is
typically in the femtosecond range.  Since thermal vibrations are on the
picosecond time-scale, each x-ray absorbed in an EXAFS measurement gives a
``snapshot'' of the structure around 1 randomly selected absorbing atom in
the sample, and the neighboring atoms will be essentially frozen in some
configuration.  Building up a full spectrum will result in a ``blurry
picture'' due to the addition of many (often billions) of these snapshots.
This has the important consequence that a single EXAFS measurement cannot
distinguish thermal disorder due to atomic vibrations from static disorder.

An EXAFS measurement is then a {\emph{sampling}} of the configuration of
atoms around the average absorbing atom.  This configuration is called the
Partial Pair Distribution function, $g(R)$, which gives the probability that
an atom is found a distance $R$ away from an atom of the selected type.
Pair distribution functions are found from many structural probes (notably
scattering techniques), but the Partial aspect is unique to EXAFS and other
element-specific probes.  EXAFS is sensitive only to the pairs of atoms
including that absorbing atom.  Thus while scattering can give very
accurate measures of the total pair distribution function, EXAFS is
particularly useful for looking at low concentration elements in complex
systems.

To better account for the sampling of $g(R)$, we should replace our
$\sigma^2$ term with an integral over {\emph{all}} absorbing atoms, as with
(using a simplified form of the EXAFS Equation in exponential notation and
recalling that $k$ might be replaced by $p$, the complex wavenumber to
account for the mean-free-path $\lambda(k)$):
\begin{equation}
    \chi(k)  = \Biggl\langle \sum_j {\frac{f_j(k)e^{i2kR_j + i\delta_j(k)}}{kR_j^2}} \Biggr\rangle
\end{equation}
\noindent
where the angle brackets mean averaging over the distribution function:
\begin{equation*}
  \langle x \rangle = \int dR\, x \, g(R) / \int dR\, g(R)
\end{equation*}
\noindent
There are a few different approaches that can be used for modeling $g(R)$
in EXAFS.  First, one can ask what the principal moments of $g(R)$ might be.
Recognizing that $e^{i2kR}$ term (or $\sin(2kR)$
term) is the most sensitive part to small changes in $R$, and pulling out the
other terms, we have
\begin{equation}
    \chi(k)  =   \sum_j {f_j(k){\frac{e^{ i\delta_j(k)} }{kR_j^2}}}   \biggl\langle e^{i2kR_j}  \biggr\rangle
\end{equation}
\noindent
This average of an exponential term can be described by the
{\emph{cumulants}} of the distribution
$g(R)$, as
\begin{equation*}
  \biggl\langle e^{i2kR} \biggr\rangle = \exp \bigg[ \sum_{n=1}^{\infty} { \frac{(2ik)^n}{n!}}C_n  \bigg].
\end{equation*}
\noindent where the coefficients $C_n$ are the cumulants.
The cumulants of a distribution can be related to the more familiar moments
of the distribution.  The lowest order cumulants are
\begin{eqnarray*}
  C_1  &  =& \langle r \rangle \\
  C_2  &  =& \langle r^2 \rangle - \langle r \rangle^2 \\
  C_3  &  =& \langle r^3 \rangle - 3 \langle r^2 \rangle  \langle r \rangle  + 2 \langle r \rangle^3     \\
  C_4  & = & \langle r^4 \rangle - 3 \langle r^2 \rangle^2  - 4\langle r^3 \rangle \langle r \rangle
      +12  \langle r^2 \rangle  \langle r \rangle^2   - 6\langle r \rangle^4
\end{eqnarray*}
\noindent where $r= R - R_0$ and $R_0$ is the mean $R$ value of the
distribution.  $C_1$ is then simply a shift in centroid, and $C_2$ is the
mean-square-displacement, $\sigma^2$. $C_3$ and $C_4$ measure the skewness
and kurtosis for the distribution, and are 0 for a Gaussian distribution.
Because the low order terms in the cumulant expansion represent a small
modification to the Gaussian approximation and can be readily applied to
any spectrum, it is included in many analyses codes and discussed widely in
the EXAFS literature.  The skewness term, $C_3$, is sometimes found to be
important in analysis of moderately disordered systems.

Another approach to modeling complex disorder is to parametrize $g(R)$ by
some functional form and use this parametrization in the EXAFS Equation.
This can be done either analytically by putting in a functional form for
$g(R)$\cite{gnxas}, or by building a histogram with weights given by the
parametrized $g(R)$.  This approach can be readily done with existing
analysis tools, and can give noticeably better results than the cumulant
expansion for very high disorder.  For some problems, a more sophisticated
analysis using a {\emph{Monte Carlo}} approach of calculating the EXAFS for
a large set of atomic clusters can be useful.  For example, atomic
configurations from a series of molecular dynamics simulations may be used
to predict EXAFS spectra including complex configurations and disorder.
Such work can be computationally expensive, but can also give additional
insight into the interactions between atoms and molecules in complex
systems.  We'll continue to use $N$ and $\sigma^2$ as the normal form of
the EXAFS Equation, but will remember that these more complex descriptions
of the distribution of atoms are possible and that we are not limited to
studying well-behaved systems with Gaussian distributions.


\subsection{Polarization}


X-ray emitted by synchrotron sources are highly polarized in the horizontal
plane, unless specifically altered by the source or X-ray optics.   The
emitted photo-electron preserves this polarization.  This can have
implications for EXAFS from ordered materials that do not have cubic
symmetry around the absorbing atom.

For $K$ shell edges, the dipole transition from the $s$ electron level means
the photo-electron goes as a $p$ wave, with electron probability having a
$\cos^2\theta$ dependence, with $\theta$ being the angle relative to the
horizontal plane.   Because of this, $K$ edge EXAFS is actually completely
blind to scattering atoms in the direction of the X-ray beam and  in the
vertical plane, above and below the absorbing atom.   For samples that are
isotropic, randomly oriented, or with cubic symmetry, the polarization is
of no consequence, but for anisotropic systems, and surfaces, this can
either be viewed as a complication, or exploited by carefully orienting the
sample relative to the horizontal plane.  Mathematically, we simply add
a $\epsilon \cdot \hat{R}$ term to the EXAFS equation,
where $\epsilon$ is X-ray polarization vector, and $\hat{R}$ is the unit vector from
absorbing atom to scattering atom.  Our EXAFS equation  now becomes
\begin{equation}
  \chi(k) = \sum_j {
    {{ \frac{N_j e^{-2k^2\sigma_j^2}
          e^{-2R/\lambda(k)} (\epsilon \cdot \hat{R_j}) f_j(k) }{kR_j^2}}}\>
    \sin[2kR_j + \delta_j(k)] } \>
      \label{Eq:xafs_withpolar}
\end{equation}
\noindent
For  $L_{\rm{III}}$ and $L_{\rm{ II}}$ edges, as well as $M$ edges, the
polarization dependence can be more complex than the simple $\cos^2\theta$
dependence of $K$ shells due to the fact that these edge have multiple
allowed angular momentum states for the outgoing photo-electron.

\subsection{Discussion}


We've used a simple physical picture of photo-electron scattering to
develop the EXAFS equation that we can use in the quantitative analysis of
EXAFS spectra,
\begin{equation}
  \chi(k) = \sum_j {
    {{\frac{S_0^2 N_j e^{-2k^2\sigma_j^2} e^{-2R_j/\lambda(k)}
 (\epsilon \cdot \hat{R_j}) f_j(k)}{kR_j^2}}}\>
    \sin[2kR_j + \delta_j(k)] }.
  \label{Eq:xafs_final}
\end{equation}
\noindent
From this, our final version of the EXAFS equation, we can draw a
few physical conclusions about XAFS.  First, because of the $\lambda(k)$
term and the $R^{-2}$ term, XAFS is seen to be an inherently \emph{local
  probe}, not able to see much further than 5 {\AA} or so from the
absorbing atom.   Second, the XAFS oscillations consist of different
frequencies that correspond to the different distances of atomic shells.
This will lead us to use Fourier transforms in the analysis.  Finally, in
order to extract the distances and coordination numbers, we need to have
accurate values for the scattering amplitude and phase-shifts $f(k)$ and
$\delta(k)$ from the neighboring atoms.

This last point here -- the need for accurate scattering amplitude and
phase-shifts -- has been a crucial issue in the field of EXAFS.  Though
early attempts to calculate the terms were qualitatively successful and
instructive, they were generally not accurate enough to be used in
analysis.  In the earliest EXAFS analyses, these factors were most often
determined from experimental spectra in which the near-neighbor distances
and species were known.  Such experimental standards can be quite accurate,
but are generally restricted to first neighbor shell.  Since the 1990s,
calculations of $f(k)$ and $\delta(k)$ have become more accurate and
readily available, and use of experimental standards in EXAFS analysis is
now somewhat rare, and restricted to studies of small changes in distances
of fairly well-characterized systems.  Calculated scattering factors are not
without problems, but they have been shown numerous times to be accurate
enough to be used in real analysis, and in some cases are more accurate
than experimentally derived scattering factors.  The calculated factors are
not restricted to the first shell, and can account for multiple-scattering
of the photo-electron.  In section~{\ref{sect:modeling}}, we'll use
calculations of $f(k)$ and $\delta(k)$ from {\scshape{feff}} to model real
data.


