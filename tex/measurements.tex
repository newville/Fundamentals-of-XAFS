
\section{XAFS Measurements: Transmission and Fluorescence}

XAFS requires a very good measure of $\mu(E)$.  Since the XAFS is a fairly
small modulation of the total absorption, a fairly precise and accurate
measurement of $\mu(E)$ -- typically to $10^{-3}$ -- is required.
Statistical errors in $\mu(E)$ due to insufficient count rates in
intensities are rarely the limiting factor for most XAFS measurements, and
can generally be overcome by counting longer.  On the other hand,
systematic errors in $\mu(E)$ can degrade or even destroy the XAFS, and are
more difficult to eliminate.  Fortunately, if care is taken in sample
preparation, setting up the measurement system, and alignment of the sample
in the beam, it is usually not too difficult to get good XAFS measurements.

A sketch of the basic experimental layout in Figure~\ref{Fig:EXPT:cartoon},
showing a monochromatic beam of X-rays striking a sample and the
intensities of the incident, transmitted, and emitted X-ray beams being
measured.  From this, it can be seen that the main experimental challenges
are 1) getting an X-ray source that can be reliably and precisely tuned to
select a single X-ray energy, and 2) high-quality detectors of X-ray
intensity.  For most modern experiments, the X-ray source is a synchrotron
radiation source, which provides a highly collimated beam of X-rays with a
broad range of energies.  A particular energy is selected with a double
crystal monochromator, which consists of two parallel nearly perfect
crystals, typically silicon.  The first crystal is centered in the incident
X-ray beam from the source and rotated to a particular angle so as to
reflect a particular energy by X-ray diffraction following Bragg's law.  By
using near-perfect crystals, the diffracted beam is very sharply defined in
angle and so also has a very narrow energy range, and the reflectivity is
near unity.  The second crystal, with the same lattice spacing, is rotated
together with the first crystal, and positioned to intercept the diffracted
beam and re-diffract so that it is parallel to the original X-ray beam,
though typically offset vertically from it.  Such a monochromator allows a
wide energy range of monochromatic X-rays to be selected simply by rotating
a single axis, and a widely used at synchrotron beamlines, and especially
at beamlines designed for XAFS measurements.

\begin{Nfig}{width=4.25truein}{exp_layout}
  \caption{Sketch of an XAFS Experiment.  An X-ray source, typically using
    synchrotron radiation, produces a collimated beam of x-rays with a
    broad energy spectrum.  These X-rays are energy-selected by a slit and
    monochromator.  The incident X-ray intensity, $I_0$, is sampled.  XAFS
    can be recorded by measuring the intensity transmitted through the
    sample or by measuring the intensity of a secondary emission --
    typically X-ray fluorescence or Auger electrons resulting from the
    X-ray absorption.  The X-ray energy is swept through and above the
    electron binding energy for a particular energy level of the element of
    interest.}
  \label{Fig:EXPT:cartoon}
\end{Nfig}


The principle characteristics of a monochromator that are important for
XAFS are {\emph{the energy resolution}}, the reproducibility, and the
stability of the monochromator. Energy resolutions of $\approx 1\rm\, eV$
at 10 keV are readily achieved with silicon monochromators using the
Si(111) reflection, and are sufficient for most XAFS measurements.  Higher
resolution can be achieved by using a higher order reflection, such as
Si(220) or Si(311), but this often comes at a significant loss of
intensity.  In addition, the angular spread of the incident X-ray beam from
the source can contribute to the energy resolution, and many beamlines
employ a reflective mirror that can be curved slightly to collimate the
beam before the monochromator to improve resolution.  While poor energy
resolution can be detrimental to XAFS measurements, and especially for
XANES measurements, most existing beamlines have resolution sufficient for
good XAFS measurements.

Stability and reproducibility of monochromators is sometimes challenging,
as the angular precisions of monochromators needed for XAFS are typically
on the order of $10^{-4}$ degrees, so that a very small change in Bragg
angle corresponds to a substantial energy shift.  Very high quality
rotation stages can essentially eliminate such drifts, but may not be
installed at all beamlines.  In addition, small temperature drifts of the
monochromator can cause energy drifts, as the lattice constant of the
crystal changes.  Stabilizing the temperature of the monochromator is very
important, but can be challenging as the power in the white X-ray beam from
a modern synchrotron source can easily exceed 1 kW in a few square
millimeters.  For the most part, these issues are ones of beamline and
monochromator design and operation, generally solved by the beamline, and
are not a significant problem at modern beamlines designed for XAFS
measurements.  Still, these issues are worth keeping in mind when assessing
XAFS data.

Despite their name, monochromators based on Bragg diffraction do not select
only one energy (or color) of light, but also certain {\emph{harmonics}}
(integer multiplies) of that energy.  While these higher energies will be
far above the absorption edge, and so not be absorbed efficiently by the
sample, they can cause subtle problems with the data that can be hard to
diagnose or correct afterward.  These include sharp changes or
{\emph{glitches}} in intensity at particular energies, and unexpectedly
large noise in the data.  There are two main strategies for removing
harmonics.  The first is to slightly misalign or ``de-tune'' the two
crystals of the monochromator.  This will reduce the transmitted intensity
of the higher-energy harmonics much more than it reduces the intensity of
the fundamental beam.  De-tuning in this way can be done dynamically, often
by putting a small piezo-electric crystal on the second monochromator
crystal to allow fine motions to slightly misalign the two crystals.  The
second method for removing harmonics is to put a reflective X-ray mirror in
the beam so that it reflects the fundamental beam but not the higher energy
harmonics.  Such a harmonic-rejection mirror is generally more efficient at
removing the higher harmonics than de-tuning the monochromator crystals.
Ideally, both of these strategies can be used, but it is generally
necessary to use at least one of these approaches.

Having linear detectors to measure $I_0$ and $I$ for transmission
measurements is important for good XAFS measurements, and not especially
difficult.  A simple ion chamber (a parallel plate capacitor filled with an
inert gas such nitrogen or argon, and with a high voltage across it through
which the X-ray beam passes) is generally more than adequate, as these
detectors themselves are generally very linear over a wide range of X-ray
intensities.  The currents generated from the detectors are quite low
(often in the picoampere range, and rarely above a few microampere) and so
need to be amplified and transmitted to a counting system.  Noise in
transmission lines and linearity of the amplification systems used for ion
chambers (and other detectors) can cause signal degradation, so keeping
cables short and well-grounded is important.  Typical current amplifiers
can have substantial non-linearities at the low and high ends of their
amplification range, and so have a range of linearity limited to a few
decades.  For this reason, significant dark currents are often set and one
must be careful to check for saturation of the amplifiers.  In addition,
one should ensure that the voltage applied across the ion chamber plates is
sufficiently high so that all the current is collected -- simply turning up
the voltage until the intensity measured for a incident beam of constant
intensity is itself constant and independent of voltage is generally
sufficient.  Such checks for detector linearity can be particularly
important if glitches are detected in a spectrum.  For fluorescence
measurements, several kinds of detectors can be used in addition to ion
chambers, and linearity can become an important issue and depend on details
of the detector.

With a good source of monochromatic X-rays and a good detection system,
accurate and precise transmission measurements on uniform samples of
appropriate thickness, are generally easy.  Some care is required to make
sure the beam is well-aligned on the sample and that harmonics are not
contaminating the beam, but obtaining a noise level of $10^{-3}$ of the
signal is generally easy for transmission measurements.  Such a noise level
is achievable for fluorescence measurements but can be somewhat more
challenging, especially for very low concentration samples.

\subsection{Transmission XAFS measurements}

For concentrated samples, in which the element of interest is a major
component -- 10\% by weight or higher is a good rule of thumb -- XAFS
should be measured in transmission.  To do this, one needs enough
transmission through the sample to get a substantial signal for $I$.  With,
${\mu}t=\ln(I/I_0)$, we typically adjust the sample thickness $t$ so that
${\mu}t \approx 2.5$ above the absorption edge and/or the edge step $\Delta
\mu(E)t \approx 1$.  For Fe metal, this gives $t=7\, \rm {\mu}m$, while for
many solid metal-oxides and pure mineral phases, $t$ is typically in the
range of 10 to 25 $\rm {\mu}m$.  For concentrated solutions, sample
thickness may be several millimeter thick, but this can vary substantially.
If both $\mu t \approx 2.5$ for the total absorption and an edge step
$\Delta \mu(E)t \approx 1$ cannot be achieved, it is generally better to
have a smaller edge step, and to keep the total absorption below $\mu t
\approx 4$.  Tabulated values for $\mu(E)$ for the elements are widely
available, and software such as {\scshape{hephaestus}}\cite{horae} can
assist in these calculations.

In addition to requiring the right thickness for transmission measurements,
the sample must be of uniform thickness and free of pinholes.
Non-uniformity (that is, variations in thickness of a factor of 2 or so)
and pinholes in the sample can be quite damaging, as $\mu$ is logarithmic
in $I$.  Since the portion of the beam going through a small hole in the
sample will transmit with very high intensity, it will disproportionately
contribute to $I$ compared to the parts of the beam that actually go
through the sample.  For a powder, the grain size cannot be much bigger
than an absorption length, or this too will lead to non-linear variations
in the beam transmitted through the sample.  If these challenging
conditions can be met, a transmission measurement is very easy to perform
and gives excellent data.  This method is usually appropriate for pure
mineral phases, or for other systems in which the absorbing element has a
concentrations $>$ 10\%.

A few standard methods for making uniform samples for transmission XAFS
exist.  If one can use a solution or has a thin, single slab of the pure
material (say, a metal foil, or a sample grown in a vacuum chamber), these
can make ideal samples.  For many cases, however, a powder of a reagent
grade chemical or mineral phase is the starting material.  Because the
required total thickness is so small, and uniformity is important, grinding
and sifting the powder to selected the finest grains can be very helpful.
Using a solvent or other material in the grinding process can be useful.  In
some case, suspending a powder in a solvent to skim off the smallest
particles held up by surface tension can also be used.  Spreading or
painting the grains onto sticky tape and shaking off any particles that
don't stick can also be used to select the finest particles, and can make a
fairly uniform sample, with the appropriate thickness built up by stacking
multiple layers.  Ideally, several of these techniques can be used in
combination.

\begin{Nfig}{width=3.5truein}{fluor_spectra}
  \caption{X-ray fluorescence spectrum from an Fe-rich mineral (a
    feldspar), showing the Fe $K_{\alpha }$ and $K_{\beta }$ emission lines
    around 6.4 and 7.0 keV, and the elastically (and nearly-elastically)
    scattered peak near 8.5 keV. At lower energies, peaks for Ca, Ti, and V
    can be seen.}
  \label{Fig:EXPT:fluor}
\end{Nfig}

\subsection{Fluorescence XAFS measurements}

For samples that cannot be made thin enough for transmission or with the
element of interest at lower concentrations (down to the ppm level and
occasionally lower), monitoring the X-ray fluorescence is the preferred
technique for measuring the XAFS.  In a fluorescence XAFS measurement, the
X-rays emitted from the sample will include the fluorescence line of
interest, fluorescence lines from other elements in the sample, and both
elastically and inelastically (Compton) scattered X-rays.  An example
fluorescence spectrum is shown in Figure~\ref{Fig:EXPT:fluor}.  This shows Fe
$K_{\alpha}$ and $K_{\beta}$ fluorescence lines along with the elastically
scattered peak (unresolvable from the Compton scatter), as well as
fluorescence lines from Ca, Ti, and V.  In many cases the scatter or
fluorescence lines from other elements will dominate the fluorescence
spectrum.

There are two main considerations for making good fluorescence XAFS
measurements: the solid angle collected by the detector, and the energy
resolution for fluorescence lines. The need for solid angle is easy to
understand.  The fluorescence is emitted isotropically, and we'd like to
collect as much of the available signal as possible.  X-rays that are
elastically and inelastically scattered (for example, by the Compton
scattering process) by the sample are not emitted isotropically because the
X-rays from a synchrotron are \emph{polarized} in the plane of the
synchrotron, (a fact we've neglected up to this point).  This polarization
means that elastic scatter is greatly suppressed at $90^{\circ}$ to the
incident beam, in the horizontal plane.  Therefore, fluorescence detectors
are normally placed at a right angle to the incident beam.

Energy resolution for a fluorescence detector can be important as it allows
discrimination of signals based on energy, so that scattered X-rays and
fluorescence lines from other elements can be suppressed relative to the
intensity of the fluorescence lines of interest.  This lowers the
background intensity, and increases the signal-to-noise level.  Energy
discrimination can be accomplished either physically, by filtering out
unwanted emission before it gets to the detector, or electronically after
it is detected, or both.

\begin{Nfig}{width=3.5truein}{filtered_spectra}
  \caption{The effect of a ``Z-1'' filter on a measured fluorescence
    spectrum.  A filter of Mn placed between sample and detector will absorb
    most of the scatter peak, while transmitting most of the Fe
    $K_{\alpha}$ emission. For samples dominated by the scatter peak, such
    a filter can dramatically improve the signal-to-noise level.}
  \label{Fig:EXPT:filt}
\end{Nfig}

An example of a commonly used physical filter is to place a Mn-rich
material between an Fe-bearing sample and the fluorescence detector.  Due
to the Mn $K$ absorption edge, the filter will preferentially absorb the
elastic and inelastic scatter peak and pass the Fe $K_{\alpha }$ line, as
shown in Figure~\ref{Fig:EXPT:filt}. For most $K$ edges, the element with
$Z-1$ of the element of interest can be used to make an appropriate filter,
and appropriate filters can be found for most of the $L$ edges.  A simple
filter like this can be used with a detector without any intrinsic energy
resolution, such as an ion chamber. To avoid re-radiation from the filter
itself, Soller slits, as shown in Figure~\ref{Fig:EXPT:soller}, can be used
to preferentially collect emission from the sample and block any signal
generated away from the sample from getting into the fluorescence detector,
including emission from the fiter itself.  Such an arrangement can be very
effective especially when the signal is dominated by scatter, as when the
concentration of the element of interest is in the range of hundreds of ppm
or lower.

\begin{Nfig}{width=2.0truein}{soller}
  \caption{The practical use of ``Z-1'' filter for energy discrimination of
    a fluorescence spectrum.  The filter placed between sample and detector
    will absorb most of the scatter peak.  Because the filter can itself
    re-radiate, a set of metal Soller slits pointing at the sample will
    preferentially absorb the emission from the filter and prevent it from
    entering the detector.}
  \label{Fig:EXPT:soller}
\end{Nfig}

Energy discrimination can also be done electronically on the measured X-ray
emission spectrum after it has been collected in the detector.  A common
example of this approach uses a solid-state Si or Ge detector, which can
achieve energy resolutions of a $\approx 200\, \rm{ eV}$ or better.  The
spectrum shown in Figure~\ref{Fig:EXPT:fluor} was collected with such a Ge
solid-state detector.  These detectors have an impressive advantage of
being able to measure the full X-ray fluorescence spectrum, which is useful
in its own right for being able to identify and quantify the concentrations
of other elements in the sample.  Because unwanted portions of the
fluorescence spectrum can be completely rejected electronically, these
detectors can have excellent signal-to-background ratios and be used for
XAFS measurements with concentrations down to ppm levels.  Though
solid-state detectors have many advantages, they have a few drawbacks:

\begin{description}
\item[Dead time:] The electronic energy discrimination takes a finite
  amount of time, which limits the total amount of signal that can be
  processed.  These detectors typically saturate at $\approx 10^{5}\, \rm{
    Hz}$ of {\emph{total}} count rate or so.  When these rates are
  exceeded, the detector is effectively unable to count all the
  fluorescence, and is said to be ``dead'' for some fraction of the time.
  It is common to use ten or more such detectors in parallel.  Even then,
  the limit on total intensity incident for these detectors can limit the
  quality of the measured XAFS.  This will be discussed more below.

\item[Complicated:] Maintaining, setting up, and using one of these is much
  more work than using an ion chamber.  For example, germanium solid-state
  detectors must be kept at liquid nitrogen temperatures.  The electronics
  electronics needed for energy discrimination can be complicated,
  expensive, and delicate.
\end{description}
\noindent
Despite these drawbacks, the use of solid-state detectors is now fairly
common practice for XAFS, especially for dilute and heterogeneous samples,
and the detectors and electronics themselves are continually being
improved.

Before we leave this section, there are two import effect to discuss for
XAFS measurements made in fluorescence mode.  These are
\emph{self-absorption} or over-absorption from the sample, and a more
detailed explanantion of deadtime effects for measurements made with
solid-state detectors.  If not dealt with properly, these effects can
substantially comprise otherwise good XAFS data, and so it is worth some
attention to understand these in more detail.

\subsection{Self-Absorption (or Over-Absorption) of Fluorescence XAFS}

The term \emph{self-absorption} when referred to fluorescence XAFS can be
somewhat confusing.  Certainly, the sample itself can absorb many of the
fluoresced X-rays.  For example for a dilute element (say, Ca) in a
relatively dense matrix (say, iron oxide), the Ca fluorescence will be
severely attenuated by the sample and the measured fluorescence signal for
Ca will be dictated by the escape depth of the emitted X-ray in the matrix.

Though that is an important consideration, and the meaning of the term
{\emph{self-absorption}} in quantitative X-ray fluorescence analysis, this
is not what is usually meant by the term in EXAFS.  Rather, the term
self-absorption for EXAFS usually refers to the case where the penetration
depth into the sample is dominated by the element of interest, and so is
one special case of the term as used in X-ray fluorescence analysis.  In
the worst case for self-absorption (a very thick sample of a pure element),
the XAFS simply changes the penetration depth into the sample, but
essentially all the X-rays are absorbed by the element of interest.  The
escape depth for the fluoresced X-ray is generally much longer than the
penetration depth, so that most absorbed X-rays will generate a fluoresced
X-ray that will escape from the sample.  This severely dampens the XAFS
oscillations, and for a very concentrated sample, there may be no XAFS
oscillations at all.  With this understanding of the effect, the term
{\emph{over-absorption}}\cite{ManceauRIMG2002} is probably a better
description, and should be preferred to {\emph{self-absorption}} even
though the latter is in more common usage.

\begin{Nfig}{width=2.0truein}{self_absorb}
  \caption{fluorescence X-ray absorption measurements, showing incident
    angle $\theta$ and exit angle $\phi$. }
  \label{Fig:MEAS:expt1}
\end{Nfig}


Earlier we said that for XAFS measured in fluorescence
\begin{equation}
  \mu (E) \propto I_{f}/I_{0}.
\end{equation}
\noindent
This is a slight oversimplification. The probability of fluorescence is
proportional to the absorption probability but the fluorescence intensity
that we measure has to travel back through the sample to get to the
detector. Since all matter attenuates X-rays, the fluorescence intensity,
and therefore the XAFS oscillations, can be damped.  More correctly, the
measured fluorescence intensity goes as (see Fig.~\ref{Fig:MEAS:expt1})

\begin{equation}
  I_f = I_0 {\frac{\epsilon\Delta\Omega}{4\pi}} \,
  {\frac{\mu_{\chi}(E) \big\{ 1- e^{-[\mu_{\rm tot}(E)/\sin\theta +
        \mu_{\rm tot}(E_f)/\sin\phi]t}\big\}}{\mu_{\rm tot}(E)/\sin\theta
      +\mu_{\rm tot}(E_f)/\sin\phi}}
  \label{EQ:selfabs}
\end{equation}
\noindent
where $\epsilon$ is the fluorescence efficiency, $\Delta\Omega$ is the
solid angle of the detector, ${E_f}$ is the energy of the fluorescence
X-ray, $\theta$ is the incident angle (between incident X-ray and sample
surface), $\phi$ is the exit angle (between fluoresced X-ray and sample
surface), ${\mu_{\chi}(E)}$ is the absorption from the
element of interest,  and ${\mu_{\rm tot}(E)}$ is the {\emph{total}}
absorption in the sample,
\begin{equation}
 \mu_{\rm tot}(E) = \mu_{\chi}(E) + \mu_{\rm other}(E)
\end{equation}

Eq.~\ref{EQ:selfabs} has several interesting limits that are common for
real XAFS measurements.  First, there is the {\emph{thin sample limit}},
for which $\mu t \ll 1$.  The $ 1 - e^{-\mu t} $ term then becomes
(by a Taylor series expansion)
\begin{equation*}
 1 - e^{-\mu t} \approx [{\mu_{\rm tot}(E)/\sin\theta +\mu_{\rm
     tot}(E_f)/\sin\phi}]\, t
\end{equation*}
\noindent which cancels the denominator, so that

\begin{equation}
  I_f \approx I_0 {\frac{\epsilon\Delta\Omega}{4\pi}}
{\mu_{\chi}(E)} t
\end{equation}

Alternatively, there is the {\emph{thick, dilute sample limit}}, for which
$\mu t \gg 1$ and $\mu_\chi \ll \mu_{\rm other}$.  Now the exponential term
goes to 0, so that
\begin{equation}
  I_f = I_0 {\frac{\epsilon\Delta\Omega}{4\pi}} \,
  {\frac{\mu_{\chi}(E)}{\mu_{\rm tot}(E)/\sin\theta
      +\mu_{\rm tot}(E_f)/\sin\phi}}    .
\end{equation}
\noindent
We can then ignore the energy  dependence of $\mu_{\rm tot}$, leaving
\begin{equation}
  I_f \propto I_0 \mu_{\chi}(E)
\end{equation}
\noindent
These two limits (very thin or thick, dilute samples) are the best cases
for fluorescence measurements.

For relatively thick, concentrated samples, for which $\mu_\chi \approx
\mu_{\rm other}$, so that $ \mu_\chi \approx \mu_{\rm tot} $ we cannot ignore
the energy dependence of $\mu_{\rm tot}$, and must correct for the
oscillations in $\mu_{\rm tot}(E)$ in Eq.~\ref{EQ:selfabs}. As said above,
for very concentrated samples, $ \mu_{\rm tot}(E) \approx \mu_{\chi}(E)$,
and the XAFS can be completely lost.  On the other hand, if the
self-absorption is not too severe, it can be corrected using the above
equations\cite{BoothBridges2005,Pfalzer1999}.

Finally, these self-absorption effects can be reduced for thick,
concentrated samples by rotating the sample so that it is nearly normal to
the incident beam.  With $\phi \rightarrow 0$ or the {\emph{grazing exit
    limit}}, $\mu_{\rm tot}(E_f)/\sin\phi \gg \mu_{\rm tot}(E)/\sin\theta$,
giving
\begin{equation}
  I_f \approx I_0 {\frac{\epsilon\Delta\Omega}{4\pi}} \,
  {\frac{\mu_{\chi}(E)}{\mu_{\rm tot}(E_f)/\sin\phi}}
\end{equation}
\noindent
which gets rid of the energy dependence of the denominator.


In certain situations, monitoring the intensity of emitted electrons (which
includes both Auger electrons and lower-energy secondary electrons) can be
used to measure the XAFS. The escape depth for electrons from material is
generally much less than a micron, making these measurements more
surface-sensitive than X-ray fluorescence measurements, and essentially
immune to over-absorption.  These electron yield measurements are generally
most appropriate for samples that are metallic or semiconductor (that is,
electrically conducting enough so that the emitted electrons can be
replenished from a connection to ground, without the sample becoming
charged).  For these reasons, measuring the XAFS in electron yield is not
very common, and details of these measurements will be left for further
reading.


\subsection{Deadtime Corrections for Fluorescence XAFS }

For fluorescence XAFS data measured with an energy discriminating
fluorescence detector, such as a solid-state Ge or Si detector, it is often
necessary to correct for the so-called {\emph{deadtime effect}}.  This
accounts for the fact that a finite amount of time is needed to measure the
energy of each X-ray detected, and the electronics used to make this
measurement can only process one photon at at time.  At high enough
incident count rates, the detector electronics cannot process any more
counts and is said to be {\emph{saturated}}.  These effects can be
particularly important when the absorbing atom is of relatively high
concentration (above 1 \% by weight or so), because the intensity of the
monitored fluorescence line is negligible below the edge, and grows
dramatically at the absorption edge.  This can add a non-linear reduction
of the fluorescence intensity, and so give non-linear artifact to the EXAFS
and XANES.

\begin{Nfig}{width=3.0truein}{deadtime}
  \caption{Typical deadtime curve for a pulse-counting,
    energy-discriminating detector with a deatime $\tau$ of 2 $\rm{\mu}s$.
    At low input count rate, the output count rate -- the rate of
    successfully processed data -- rises linearly.  As the count rate
    increases, some of the pulses cannot be processed, resulting in a
    reduced output count rate lower.  At {\emph{saturation}}, the output
    count rate cannot go any higher, and increasing the input count rate
    will decrease the output rate.  The dashed line shows a line with unity
    slope, for a detector with no deadtime.}
  \label{Fig:RED:deadtime}
\end{Nfig}

Fortunately, most energy discriminating detector and electronics systems
can be characterized with a simple parameter $\tau$ that relates the
incident count rate with the output count rate actually processed as
\begin{equation}
I_{\rm out} = I_{\rm inp} e^{( - I_{\rm inp} \tau)}
\end{equation}
\noindent where $I_{\rm inp} $ is the incident count rate to the detector,
$I_{\rm out} $ is the output count rate, giving the intensity reported by
the detector, and $\tau$ is the deadtime, characteristic of the detector
and electronics system.  For a realistic value of $\tau = 2{\rm\,\mu s}$,
the relation of input count rate and output count rate is shown in
Figure~\ref{Fig:RED:deadtime}.  For many detector systems, there is some
ability to adjust the maximum output count rate, and so $\tau$, that can be
achieved, at the expense of energy resolution of the fluorescence spectra.
Of course, to make this correction, one wants to get $I_{\rm inp}$ given
$I_{\rm out}$ which can be complicated for very high count rates.  For some
detector systems, one can simply record $I_{\rm inp}$ and $I_{\rm out}$ for
each measurement as an output of the detector and electronics system.
Alternatively, one can separately measure $\tau$ so that the corrections
can be applied easily.  Otherwise, a good rule of thumb is that spectra can
be corrected up to a rate for which $I_{\rm out}$ is half of $I_{\rm inp}$
($I_{\rm inp}$ around 350 kHz for the curve shown in
Figure~\ref{Fig:RED:deadtime}).  Importantly, for multi-element detector
systems, each detector element will have its own deadtime, and corrections
should be made for each detector before summing the signals from multiple
detectors.
