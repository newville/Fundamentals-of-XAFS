\documentclass{article}
\usepackage{tikz}
\usepackage{pgfplots}
\usepackage{verbatim}
\usepackage[active,tightpage]{preview}
\PreviewEnvironment{tikzpicture}
\setlength\PreviewBorder{3pt}

\usetikzlibrary{decorations.pathmorphing, arrows, intersections, snakes}

\pgfmathdeclarefunction{gauss}{2}{%
  \pgfmathparse{1/(#2*sqrt(2*pi))*exp(-((x-#1)^2)/(2*#2^2))}}


%%
%% \ifscatter will control whether the scattering atom is displayed
%%
\newif\ifscatter

\scattertrue  %% for absorption with scattering
\scatterfalse %% for bare absorbing atom version

\begin{document}
\begin{tikzpicture}[scale=0.25,  >=stealth, inner sep=0pt, outer sep=2pt,%
  axis/.style={thick,->}, wave/.style={thick,color=#1,smooth}]

  \begin{scope} % black lines
    \draw (6, 0) -- ++ (0, 25)  node[above] {absorbing atom} ;
    
    \ifscatter
       \draw (18, 0) -- ++ (0, 25) node[above] {scattering atom}  ;
    \fi
        
    \draw (0, 9) -- ++ (36.25,  0) node[right=2, below=0.25, rotate=90] {$E_{0}$};
    \draw[->,>=latex, thick] (36, 0) -- ++(0, 25) 
         node[sloped,right=0,below=2, midway] {Energy};

    \draw[->,>=latex, thick] (36, 0) -- ++(-13, 0) 
         node[sloped,right=0,below=7, rotate=90, midway] {$\mu$};
  \end{scope}

  \begin{scope}[color=black!90!green!35]
    \draw[very thick, >=latex] (0, 8.95) to [out=0, in=92] (5.75, 0) ;
    \draw[very thick, >=latex] (12, 8.95) to [out=180, in=88] (6.25, 0) ;
    
    \ifscatter
        \draw[very thick, >=latex] (12, 8.95) to [out=0, in=92] (17.75, 0) ;
        \draw[very thick, >=latex] (24, 8.95) to [out=180, in=88] (18.25, 0) ;
    \fi
  \end{scope}

  \begin{axis}[xshift=50mm, yshift=-2mm, no markers, domain=0:20, samples=100,
              height=36mm, width=36mm, axis x line=none, axis y line=none,
              xtick=\empty, ytick=\empty]
    \addplot [line width=4pt, blue!80!black] {gauss(10,1.0)};
  \end{axis}

  \begin{scope}[color=red!50!black]
    \draw (12, 21) node {{\small{$\lambda \sim 1/\sqrt{(E - E_0)}$}}} ++ (0, 0);
    \draw (12, 23) node {{\small{photo-electron}}} ++ (0, 0);

    \draw[very thick, variable=\x, samples at={0,0.08,...,12}] 
         plot (\x+6, {11+ 16*cos(1.0*\x r)/ (\x+15)})
         plot (\x+6, {14+ 16*cos(2.9*\x r)/ (\x+15)})
         plot (\x+6, {17+ 16*cos(4.1*\x r)/ (\x+15)});
    
     \draw[very  thick] plot [smooth] coordinates  {( 33, 1) (35, 8.0) 
       (34, 8.88) (27, 9.1) (27.7, 9.9)  (28.5, 14) (31, 25) } ;
   \end{scope}

   \begin{scope}[color=blue!70!black]
     \draw[->, thick, >=stealth', shorten >=-5pt, decorate, 
          decoration={snake, amplitude=2.5}] 
          (-1, 3.) --  node[above=4, midway] {{\small{X-ray}}} (4.5, 0.5);
     \draw (6.7, 0.7) node[right] {$1s$ core level};
     \ifscatter
        \draw[ very thick] plot [smooth] coordinates {
          (27.2, 9.1) (26.1, 9.2) (28.5, 10.) (27.3, 10.6) (27.2, 10.8)
          (28.4, 11.4)  (27.7, 12.2) (28.6, 13.2) (28.3, 14.3) 
          (29, 16)  (29.5, 18) (29.8, 20)  (31, 25) } ;
         \draw[very thick, variable=\x, samples at={0,0.08,...,12}]  
              plot (\x+6, {11+ 12*cos((1.7 + 1.0*\x) r)  / (30-\x)})
              plot (\x+6, {14+ 12*cos((0 + 2.9*\x) r) / (30-\x)})
              plot (\x+6, {17+ 12*cos((2.8 + 4.1*\x) r) / (30-\x)});
      \fi
    \end{scope}

  \end{tikzpicture}
\end{document}

